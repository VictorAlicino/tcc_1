%-----------------------------------------------------------------
% UNIOESTE - Ciência da Computação
% 4o. ano - Trabalho de Conclusão de Curso
% Profa. Teresinha Arnauts Hachisuca 
%-----------------------------------------------------------------
%DECLARAÇÃO DO TIPO DE DOCUMENTO, TAMANHO DA FOLHA E FONTE
%-----------------------------------------------------------------
\documentclass[
	% -- opções da classe memoir --
	12pt,				% tamanho da fonte
% 	twoside,			% para impressão em verso e anverso. Oposto a oneside
	a4paper,			% tamanho do papel. 
	% -- opções da classe abntex2 --
	%chapter=TITLE,		% títulos de capítulos convertidos em letras maiúsculas
	%section=TITLE,		% títulos de seções convertidos em letras maiúsculas
	%subsection=TITLE,	% títulos de subseções convertidos em letras maiúsculas
	%subsubsection=TITLE,% títulos de subsubseções convertidos em letras maiúsculas
	% -- opções do pacote babel --
	english,			% idioma adicional para hifenização
	brazil,				% o último idioma é o principal do documento
	]{article}
	
%-----------------------------------------------------------------
%DEFINIÇÕES DOS PACOTES UTILIZADOS
%-----------------------------------------------------------------
\usepackage{cmap}				% Mapear caracteres especiais no PDF
\usepackage{lmodern}			% Usa a fonte Latin Modern			
\usepackage[T1]{fontenc}		% Selecao de codigos de fonte.
\usepackage[utf8]{inputenc}		% Codificacao do documento (conversão automática dos acentos)
\usepackage{lastpage}			% Usado pela Ficha catalográfica
%\usepackage{indentfirst}		% Indenta o primeiro parágrafo de cada seção.
\usepackage{color}				% Controle das cores
\usepackage{graphicx}			% Inclusão de gráficos

\usepackage[brazil]{babel}      % Permite traduzir termos do LateX para português Brasil.
\usepackage{hyperref}           % Permite ativar hyperlinks
\usepackage{algorithm}          % pacote que de suporte a representação de algoritmos.

\usepackage{algorithmic}
\usepackage[pdftex]{geometry}
\usepackage{multirow}
\graphicspath{ {./images/} }
\usepackage[center]{caption}

\usepackage[resetlabels,labeled]{multibib}

% ---
% Pacotes de citações
% ---
\usepackage[brazilian,hyperpageref]{backref}	 % Paginas com as citações na bibl
\usepackage[alf]{abntex2cite}	% Citações padrão ABNT
%\citebrackets[]

% --- 
% CONFIGURAÇÕES DE PACOTES
% --- 

% ---
% Configurações do pacote backref
% Usado sem a opção hyperpageref de backref
\renewcommand{\backrefpagesname}{Citado na(s) página(s):~}

% Texto padrão antes do número das páginas
\renewcommand{\backref}{}

% Define os textos da citação
\renewcommand*{\backrefalt}[4]{
	\ifcase #1 %
		Nenhuma citação no texto.%
	\or
		Citado na página #2.%
	\else
		Citado #1 vezes nas páginas #2.%
	\fi}%
% ---

%-----------------------------------------------------------------
%               INICIO DO DOCUMENTO - CAPA
%-----------------------------------------------------------------
\begin{document}


\begin{center}

	\textsc{
		\large
			\\universidade estadual do oeste do paraná
			\\unioeste - campus de foz do iguaçu
			\\centro de engenharias e ciências exatas
			\\curso de ciência da computação
			\\[1 cm]tcc - trabalho de conclusão de curso
	}
	\\
	[4 cm]
	\large Proposta de Trabalho de Conclusão de Curso
    \\
	\textbf{
	    \textsc{Desenvolvimento de um sistema multiusuários para edifícios inteligentes}
    }
	\\[5 cm]Victor Hugo de Almeida Alicino
    \\Orientador(a): Antonio Marcos Massao Hachisuca
    \\[2 cm]Foz do Iguaçu, 12 de agosto de 2024
    
\end{center}

\thispagestyle{empty}

%-----------------------------------------------------------------
%               IDENTIFICAÇÃO DO PROJETO
%-----------------------------------------------------------------
\section{Identificação}
    
    \subsection{Área e Linha de Pesquisa} 
        \noindent Grande Área: Ciência da Computação
        \\Código: 1.03.00.00-7
    	\\[1 cm]Linha de Pesquisa: Sistemas de Computação  
    	\\Código: 1.03.04.00-2
    	\\[1 cm]Especialidade: Software Básico
    	\\Código: 1.03.04.03-7
    	
    	
    \subsection{Palavras-chave}

        \begin{enumerate}
        	\item Edifício inteligente
        	\item Cidade inteligente
        	\item Internet das Coisas
        \end{enumerate}



%-----------------------------------------------------------------
%               INTRODUÇÃO E JUSTIFICATIVA
%-----------------------------------------------------------------
\section{Introdução e Justificativa}

O tema cidades inteligente vem ganhando muita atenção nos últimos anos, tais cidades tem por estratégia gerar uma melhor 
qualidade de vida para o cidadão e um desenvolvimento econômico e social mais sustentável através do uso de tecnologias da 
informação e comunicação \cite{cetic}, um membro das cidades inteligente são os edifícios inteligentes. Indivíduos passam grande 
parte da sua vida dentro de edifícios, crescem, estudam e se desenvolvem neles \cite{art1} \cite{noauthor_how_nodate}, ao levar 
em conta a importância que essas estruturas têm no cotidiano, fica claro seu papel nas cidades inteligentes. 
Desta forma, um dos passos para a realização das cidades inteligentes é a modernização dos edifícios, alinhá-los com as propostas
estabelecidas de melhorar a qualidade de vida do ocupante e contribuir com um desenvolvimento sustentável.

Um edifício se torna inteligente através da adição de uma série de dispositivos com a finalidade de monitorar e controlar o ambiente,
podem ser citados os seguintes dispositivos:
\begin{itemize}
    \item Sensores;
    \item Atuadores;
    \item Controladores;
    \item Unidade Central;
    \item Interface;
    \item Rede;
    \item Medidor inteligente.
\end{itemize}
\cite{Morvaj2011} Suas finalidades são de operar em conjunto para a realização de tomadas de decisões, sejam elas automatizadas ou 
com interferência humana.
Tais dispositivos precisam ser gerenciados por um sistema central que atende as demandas que tornam um edifício inteligente. 

Quando o assunto são casas inteligentes, sistemas como este já são realidade e estão consolidados no mercado; Segundo a plataforma
Statista, aproximadamente 91 milhões de \emph{Smart Speakers} {(}do inglês, caixa de som inteligente, dispositivos que atuam 
duplamente como caixas de som e também como interface de interação humana com algum assistente virtual{)} foram instalados nos 
Estados Unidos no ano de 2021 \cite{statista1}, dos quais desde 2020, 60\% dos \emph{Smart Speakers} já se conectavam a plataforma
Amazon Alexa \cite{statista2}, essa porcentagem continua aumentando com 64\% dos dispositivos em 2024 \cite{statista3}.
A plataforma Amazon Alexa provê ao usuário uma maneira fácil de se conectar a vários dispositivos de
casa inteligente e controlar-los todos através de um só lugar, assim como a plataforma de código aberto Home Assistant, um sistema de
automação residencial \cite{assistant_home_nodate} que oferece várias opções de configurações e personalização e possui atualmente
mais de 350 mil instalações ativas \cite{home_assistant1}. Ambos os softwares foram criados com o indivíduo e uso individual em mente, não sendo
boas opções para o uso em edifício inteligentes onde mais de um indivíduo pode querer interagir com o sistema do edifício.

O objetivo deste trabalho é desenvolver um pedaço de uma alternativa viável para um edifício inteligente, que leve em consideração a interação
de vários indivíduos com sistema.


%-----------------------------------------------------------------
%                       OBJETIVO
%-----------------------------------------------------------------	
\section{Objetivos}
%-------------------------------------------------------------------------------
\subsection{Objetivo Geral}

Este trabalho visa desenvolver um sistema básico que demonstre a viabilidade de múltiplos usuários operarem simultaneamente 
um sistema de edifício inteligente.

%------------------------------------------------------------------------------

\subsection{Objetivos Específicos}

Para alcançar o objetivo do trabalho, é proposto uma aplicação simplificada que permita entender a viabilidade do uso de um 
sistema de edifício inteligente por diversos usuários, focando no controle das unidades de climatização.


{\bf Dentre os principais objetivos específicos destacam-se: }

\begin{itemize}
    \item Realizar uma pesquisa bibliográfica sobre edifícios inteligentes e suas demandas;
    \item Estudar sobre as aplicações da internet das coisas em um edifício inteligente;
    \item Definir requisitos para um sistema de gerenciamento de um edifício inteligente;
    \item Propor e implementar um sistema de gerenciamento para edifícios inteligentes;
\end{itemize}
%-------------------------------------------------------------------------------


%-----------------------------------------------------------------
%           PLANO DE TRABALHO E CRONOGRAMA DE EXECUÇÃO
%-----------------------------------------------------------------	
\section{Plano de Trabalho e Cronograma de Execução}
    \begin{enumerate}
        \item \textbf{Implementação do Sistema Principal}; \label{a4}
        
        \item \textbf{Testes com o Hardware}; \label{a5}
        
        \item \textbf{Testes e Validação}: Testes no sistema sobre os requisitos elicitados; \label{a6}
        
        \item \textbf{Desenvolvimento da monografia}: Desenvolvimento da monografia para apresentação \label{a7}
        
        \item \textbf{Desenvolvimento da apresentação}: Desenvolvimento do material para apresentação final  \label{a8}

    \end{enumerate}

    Na Tabela \ref{tabela:cronograma1} é apresentado o cronograma das atividades.
 
\begin{table}[ht]
    \scriptsize
    \centering
    \begin{tabular}{|l|c|c|c|c|c|c|c|c|c|}
        \hline &  \multicolumn{9}{|c|}
        {\textbf{Período}} \\ \cline{2-10}
        \textbf{Atividades}                             &Ago      &Set      &Out      &Nov      &Dez      &Jan      &Fev      &Mar      &Abr        \\ \hline \hline
        \ref{a4} - Implementação do Sistema Principal   &$\bullet$&$\bullet$&$\bullet$&$\bullet$&$\bullet$&$\bullet$&         &         &           \\ \hline
        \ref{a5} - Testes com o Hardware                &         &         &$\bullet$&$\bullet$&$\bullet$&$\bullet$&         &         &           \\ \hline
        \ref{a6} - Testes e Validação                   &         &         &         &         &         &$\bullet$&$\bullet$&         &           \\ \hline
        \ref{a7} - Desenvolvimento da monografia        &$\bullet$&$\bullet$&$\bullet$&$\bullet$&$\bullet$&$\bullet$&$\bullet$&$\bullet$&           \\ \hline
        \ref{a8} - Desenvolvimento da apresentação      &         &         &         &         &         &         &         &$\bullet$&           \\ \hline
    \end{tabular}
     \caption{Cronograma das Atividades}
    \label{tabela:cronograma1}
\end{table}
%-----------------------------------------------------------------
%                   MATERIAL E MÉTODO
%----------------------------------------------------------------- 

\section{Material e Método}
Para o desenvolvimento do sistema será usada a linguagem de programação Python; Para a aplicação do usuário será utilizado o framework JavaScript React Native;
Para o hardware, é esperado utilizar o firmware Tasmota em um microcontrolador Espressif ESP32, em caso de ser necessário fazer testes com outros microcontroladores
que não suportem o firmware Tasmota, firmwares substitutos de código aberto podem ser utilizados, como OpenBeken.

As máquinas que serão usadas para realizar o trabalho tanto na parte prática quanto na teórica serão:
\begin{itemize}
    \item Laptop HP Omen 2017 - I7 7700HQ, 8GB de RAM, GTX 1050;
    \item Desktop - I7 3770, 32GB de RAM, GTX 1650.
    \item Máquina Virtual na Oracle Cloud Infrastructure - 1 Núcleo, 1 GB de RAM
\end{itemize}

%-----------------------------------------------------------------
%               CRITÉRIOS DE AVALIAÇÃO
%-----------------------------------------------------------------
\section{Critérios de Avaliação} 
     
O sistema será considerado satisfatório se cumprir com as funcionalidades levantadas na elicitação de requisitos e produzir dados para resolver a pergunta do problema
inicial {(se é ou não é viável a utilização de múltiplos usuários simultâneos em um sistema de edifício inteligente)}.
    
%-----------------------------------------------------------------
%                       REFERÊNCIAS
%----------------------------------------------------------------- 

\renewcommand\refname{}
\section{Referências}

%Referencias bibliográficas que foram utilizadas para desenvolver a proposta de TCC.
    \vspace{-4.3em}
    \bibliography{referencias}
    
%-----------------------------------------------------------------
%                   SÍNTESE BIBLIOGRÁFICA
%----------------------------------------------------------------- 
\section{Síntese Bibliográfica}

%Referencias bibliográficas que se pretende utilizar para o desenvolvimento do trabalho.
    \vspace{-3.5em}
    \begin{thebibliography}{}

\bibitem{raissa2002}
NEVES, R. P. A. de A.; CAMARGO, A. R. \emph{Espaços Arquitetônicos de Alta Tecnologia: Os Edifícios Inteligentes}. 2002.

\bibitem{Buckman2014}
BUCKMAN, A. H.; MAYFIELD, M.; BECK, Stephen B. M. \emph{What is a smart building?} \emph{Smart and Sustainable Built Environment}, v. 3, n. 2, p. 92-109, set. 2014. DOI: 10.1108/SASBE-01-2014-0003.

\bibitem{stubbings1986}
STUBBINGS, M. \emph{Intelligent Buildings}. 1986. Contracts.

\bibitem{Powell1990}
POWELL, J. A. \emph{Intelligent Design Teams Design Intelligent Buildings}. \emph{Habitat International}, v. 14, n. 3, p. 83-94, 1990.

\bibitem{Wong2005}
WONG, J. K. W.; LI, H.; WANG, S. W. \emph{Intelligent building research: A review}. \emph{Automation in Construction}, v. 14, n. 1, p. 143-159, 2005. DOI: 10.1016/j.autcon.2004.06.001.

\end{thebibliography}
    

 
\end{document}
