\chapter{Introdução} 

% Contextualização do problema
As cidades inteligente têm como objetivo melhorar a qualidade de vida dos cidadãos e promover um desenvolvimento econômico e social mais
sustentável através do uso de tecnologias da informação e comunicação\cite{cetic}. Nesse contexto, os edifícios inteligentes têm um papel
fundamental. Indivíduos passam grande parte da sua vida dentro de edifícois, crescem, estudam e se desenvolvem neles\cite{art1}\cite{noauthor_how_nodate}.
Considerando a importância dessas estruturas no cotidiano, sua modernização torna-se essencial para a concretização das cidades inteligentes.
Assim, a adaptação dos edifícios às novas tecnologias pode melhorar a qualidade de vida dos ocupantes e contribuir para um desenvolvimento
sustentável.

Para que um edifício seja considerado inteligente, é necessária a incorporação de dispositivos capazes de monitorar e controlar o ambiente,
como sensores, atuadores e controladores\cite{Morvaj2011}.

A Internet das Coisas (IoT) tem desempenhado um papel fundamental na automação de residências 


Dispositivos de Internet das Coisas (IoT) possuem essas capacidades e tem tido um crescimento constante nos últimos anos, 
a previsão é de que a década de 2020 termine com mais de 30 bilhões de dispositivos IoT conectados, 
60\% dos quais serão utilizados pelo consumidor final\cite{statista-iot-connected-devices}, tais dispositivos têm apresentado 
uma redução significativa de custo nos últimos 10 anos\cite{article1}, isso facilitaria a sua adoção em edifícios inteligentes


% Justificativa

% Problema de Pesquisa

% Objetivo Geral

% Objetivos Específicos	

% Metodologia

% Estrutura do trabalho (explicando brevemente o que será abordado em cada seção)