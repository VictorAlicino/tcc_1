\chapter{Proposta}
\label{sec:proposta}

\section{Objetivo Geral}
Este trabalho tem como objetivo criar um sistema mínimo que possa testar
a viabilidade de um sistema de edifício inteligente com múltiplos usuários.

\section{Objetivos Específicos}
Para atingir o objetivo geral, uma pequena arquitetura foi proposta, esta 
arquitetura é composta de quatro componentes principais, sendo eles:
\begin{itemize}
    \item \textbf{Servidor Local}
    \item \textbf{Servidor Remoto}
    \item \textbf{Aplicativo Móvel}
    \item \textbf{Aplicativo Web}
\end{itemize}

\subsection{Servidor Local} 
O \emph{Servidor Local} é responsável por criar uma
camada de tradução entre dispositivos IoT de diferentes fabricantes, que muitas
vezes utilizam diferentes protocolos de comunicação. Ele se comunica com esses
dispositivos e traduz comandos recebidos pelos usuários do sistema para os dispositivos
usando seus respectivos protocolos, o \emph{Servidor Local} também é responsável manter
o que cada usuário pode acessar e controlar.

\subsection{Servidor Remoto}
O servidor remoto é responsável por autenticar e armazenar usuários, manter uma
relação de todos os \emph{Servidores Locais} conectados e disponíveis e servir como uma
ponte de acesso para um usuário acessar um \emph{Servidor Local}.

\subsection{Aplicativo Móvel}
O aplicativo móvel é a interface principal do usuário com o sistema, nele um usuário pode interagir
com os dispositivos disponíveis no \emph{Servidor Local}, como por exemplo, ligar e desligar um ar condicionado.

\subsection{Aplicativo Web}
O aplicativo web é a interface pensada para o administrador do \emph{Servidor Local}, nela o administrador pode
gerenciar os edifícios, andares e salas disponíveis para aquele \emph{Servidor Local}, além de poder adicionar 
novos dispositivos e usuários.

É esperado que ao desenvolver essa arquitetura, seja possível responder a problemática inicial.