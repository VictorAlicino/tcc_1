% Resumo em português
\begin{resumo}
    % Contextualização
    Em tempos modernos, indivíduos passam grande parte de suas vidas em edifícios, com isso, os edifícios inteligentes desempenham um papel fundamental
    nas cidades inteligentes, todavia, tornar um edifício inteligente é um desafio, pois para receberem este título, edifícios devem ser capazes de 
    monitorar e controlar o ambiente em que se encontram o que é feito por sensores e atuadores industriais muitas inacessíveis devido ao alto custo.
    Nos últimos anos, dispositivos de internet das coisas (IoT) têm se tornado acessíveis, especialmente na área de casas inteligentes, onde se 
    encontram sensores e atuadores de baixo custo, fáceis de instalar e configurar o que apresenta uma oportunidade para a automação de edifícios
    de forma acessível. No entanto, a aplicação desses dispositivos em edifícios coletivos, como escritórios e coworking, possui um desafio, pois 
    muitos deles são projetados para uso individual ou residencial, o que inviabiliza seu uso em ambientes compartilhados.
    % Objetivo
    Este trabalho teve como objetivo desenvolver um sistema que fosse capaz de habilitar o controle desses dispositivos IoT em ambientes compartilhados
    para múltiplos usuários sem que eles recebam previamente permissão de um administrador do sistema.
    % Como foi feito
    Para atingir este objetivo, foi proposto um sistema chamado \emph{Opus} composto de Servidor Local, Servidor Remoto e Aplicativo Móvel, onde o
    Servidor Local é responsável por abstrair as comunicações com dispositivos IoT de diferentes fabricantes e protocolos, o Servidor Remoto é responsável
    por manter a autenticidade dos usuários e mediar a comunicação entre os Servidores Locais e o Aplicativo Móvel, por fim, o Aplicativo Móvel
    proporciona uma interface para que usuário possam interagir com o sistema.
    % Resultados
    Os testes realizados demonstraram que o sistema foi capaz de permitir que vários usuários mesmo que não registrados para o controle de um dispositivo
    específico pudessem controlá-lo com acesso de visitante temporário, permitindo que esses dispositivos IoT sejam utilizados em ambientes compartilhados.
    % Conclusão
    Conclui-se que o sistema proposto possibilita a utilização de dispositivos IoT originalmente voltados para casas inteligentes, mais acessíveis do
    que soluções tradicionais de automação predial, como alternativa viável para a conversão de edifícios coletivos, como escritórios e coworkings,
    em edifícios inteligentes.

    \vspace{\onelineskip}
    \noindent
    \textbf{Palavras-chave}: Edifício Inteligente, Internet das Coisas, SmartOffice, Automação.
\end{resumo}
    
    % Abstract em inglês
\begin{otherlanguage}{english}
\begin{abstract}
    % Contextualization
    In modern times, individuals spend a significant part of their lives inside buildings. As a result, smart buildings play a fundamental role
    in smart cities. However, transforming a building into a smart one is a challenge, as buildings must be capable of
    monitoring and controlling their environment, which is typically done through industrial sensors and actuators that are often inaccessible
    due to their high cost.
    In recent years, Internet of Things (IoT) devices have become more accessible, especially in the smart home sector, where
    low-cost sensors and actuators are available, easy to install and configure, presenting an opportunity for affordable building automation.
    Nevertheless, applying these devices in collective buildings, such as offices and coworking spaces, poses a challenge, as
    many of them are designed for individual or residential use, making their application in shared environments unfeasible.
    % Objective
    This work aimed to develop a system capable of enabling the control of these IoT devices in shared environments
    by multiple users without requiring prior authorization from a system administrator.
    % How it was done
    To achieve this goal, a system named \emph{Opus} was proposed, composed of a Local Server, a Remote Server, and a Mobile Application, where the
    Local Server is responsible for abstracting communication with IoT devices from different manufacturers and protocols, the Remote Server is responsible
    for maintaining user authenticity and mediating communication between the Local Servers and the Mobile Application, and finally, the Mobile Application
    provides an interface for users to interact with the system.
    % Results
    The tests performed demonstrated that the system was capable of allowing several users, even those not registered to control a specific device,
    to operate it with temporary guest access, enabling the use of these IoT devices in shared environments.
    % Conclusion
    It is concluded that the proposed system allows the use of IoT devices originally designed for smart homes, which are more accessible than
    traditional building automation solutions, as a viable alternative for converting collective buildings, such as offices and coworking spaces,
    into smart buildings.

    \vspace{\onelineskip}
    \noindent
    \textbf{Keywords}: Smart Buildings, Internet of Things, SmartOffice, Automation.
\end{abstract}
\end{otherlanguage}

\newpage