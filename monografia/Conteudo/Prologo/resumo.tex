% Resumo em português
\begin{resumo}
    % Contextualização
    Em tempos modernos, indivíduos passam grande parte de suas vidas em edifícios, com isso, os edifícios inteligentes desempenham um papel fundamental
    nas cidades inteligentes, todavia, tornar um edifício inteligente é um desafio, pois para receberem este título, edifícios devem ser capazes de 
    monitorar e controlar o ambiente em que se encontram o que é feito por sensores e atuadores industriais que muitas vezes não são acessíveis.
    Nos últimos anos, dispositivos de internet das coisas (IoT) têm se tornado acessíveis, especialmente na área de casas inteligentes, onde se 
    encontram sensores e atuadores de baixo custo, fáceis de instalar e configurar o que apresenta uma oportunidade para a automação de edifícios
    de forma acessível. No entanto, a aplicação desses dispositivos em edifícios coletivos, como escritórios e coworking, possui um desafio, pois 
    muitos desses dispositivos acessíveis para casas inteligentes são projetados para uso individual ou residencial, o que inviabiliza seu uso em
    ambiente compartilhados.
    % Objetivo
    Este trabalho teve como objetivo desenvolver um sistema que fosse capaz de habilitar o controle desses dispositivos IoT em ambiente compartilhados
    para múltiplos usuários sem que eles recebam previamente permissão de um administrador do sistema.
    % Como foi feito
    Para atingir este objetivo, foi proposto um sistema chamado \emph{Opus} composto de Servidor Local, Servidor Remoto e Aplicativo Móvel, onde o
    Servidor Local é responsável por abstrair as comunicações com dispositivos IoT de diferentes fabricantes e protocolos, o Servidor Remoto é responsável
    por manter a autenticidade dos usuários e mediar a comunicação entre os Servidores Locais e o Aplicativo Móvel, a interface para os usuários 
    interagirem com o sistema.
    % Resultados
    Os testes realizados demonstraram que o sistema foi capaz de permitir que vários usuários mesmo que não registrados para o controle de um dispositivo
    específico pudessem controlá-lo com acesso de visitante temporário, permitindo que esses dispositivos IoT sejam utilizados em ambientes compartilhados.
    % Conclusão
    O sistema permite que dispositivos IoT de casas inteligentes como sensores e atuadores mais acessíveis que soluções tradicionais de automação predial
    sejam utilizados para converter edifícios coletivos como escritórios e coworking em edifícios inteligentes.

    \vspace{\onelineskip}
    \noindent
    \textbf{Palavras-chave}: Edifício Inteligente, Internet das Coisas, SmartOffice, Automação.
\end{resumo}
    
    % Abstract em inglês
\begin{otherlanguage}{english}
\begin{abstract}
    In modern times, individuals spend a significant part of their lives inside buildings. As a result, smart buildings play a fundamental role
    in smart cities. However, making a building smart is a challenge, as it requires the ability to monitor and control the environment through
    sensors and industrial actuators, which are often not accessible. In recent years, Internet of Things (IoT) devices have become more affordable,
    especially in the smart home sector, where low-cost sensors and actuators are easy to install and configure. This presents an opportunity to automate
    buildings in a more accessible way. Nevertheless, applying these devices in shared buildings, such as offices and coworking spaces, brings a
    challenge since most of these affordable smart home devices are designed for individual or residential use, making them unsuitable for shared environments.

    This work aimed to develop a system capable of enabling the control of such IoT devices in shared environments for multiple users without
    requiring prior permission from a system administrator.

    To achieve this goal, a system named \emph{Opus} was proposed, composed of a Local Server, a Remote Server, and a Mobile Application.
    The Local Server is responsible for abstracting the communication with IoT devices from different manufacturers and protocols.
    The Remote Server ensures user authentication and mediates communication between Local Servers and the Mobile Application, which serves as the user interface.

    The tests demonstrated that the system successfully allowed multiple users, even those not previously registered to control a specific device,
    to interact with it through temporary guest access. This enables IoT devices to be used in shared environments.

    The system allows smart home IoT devices, such as sensors and actuators---more affordable than traditional building automation solutions---to
    be used for converting shared buildings like offices and coworking spaces into smart buildings.

    \vspace{\onelineskip}
    \noindent
    \textbf{Keywords}: Smart Buildings, Internet of Things, SmartOffice, Automation.
\end{abstract}
\end{otherlanguage}

\newpage