\section{Tecnologias}
~\label{tecnologias}

\subsection{Tecnologias utilizadas em automação de edifícios}
Nesta sessão serão apresentados algumas tecnologias utilizadas na automação de edifícios

\subsubsection{BACnet}
Protocolo de Rede de Automação e Controle de Edifícios (Building Automation and Control Network Protocol) ou BACnet é um protocolo 
mantido e desenvolvido pela ASHRAE com o objetivo de atender as necessidades de comunicações dos sistemas de controle e automação 
dos edifícios com seu sensores e atuadores. O protocolo fornece meios para que equipamentos troquem informações não importando o 
serviço que eles realizem~\cite{ASHRAE135}. A forma com que o BACnet funciona, permite que usuários não fiquem presos a sistemas 
proprietários, criando um padrão de comunicação entre os dispositvos como sistemas de AVAC, alarmes de incêndio e etc. uma das suas 
grandes vantagens é a habilidade de se adaptar a novas tecnologias de rede~\cite{Bushby2002}.
o BACnet define um modelo de informação, criando objetos para os dispositivos conectados, esses objetos representam seus serviços, 
assim como suas entradas e saídas. Esses objetos chamados de \emph{device object} definem as propriedades do dispositivo como: nome do modelo,
fabricante, status e a lista de outros objetos BACnet associados ao dispositivo~\cite{Domingues2016}.

\subsubsection{KNX}
~\label{sec:knx}
KNX ou Konnex é um padrão aberto desenvolvido para ser utilizado em automação de edifícios ou residências~\cite{Sapundzhi2020},
da mesma forma que o BACnet, o KNX integra sensores, atuadores e controladores dentro de uma rede, os dispositivos são controlados
através de um barramento e são chamados de \emph{Bus Access Unit} ou BAU, sendo possível controlar mais de 65 mil BAUs.
No KNX, cada fabricante pode implementar sua própria especificação nos dispositivos. BAUs podem ser separadas em grupos menores
chamados de de \emph{areas} que podem ser separadas em \emph{lines} de dispositivos. As mensagens trocadas pelos dispositivos KNX
pode ser transmitida através de cabos de par trançado, cabos de energia, frequências de rádio ou internet, sendo a forma mais comum o
KNX.TP que usa os cabos de par trançado~\cite{Kraus2020}.
Um ponto chave do KNX é seu método de comunicação, pouco similar ao MQTT, o KNX utiliza o padrão \emph{observador} na hora de trocar informações,
neste padrão múltiplos dispositivos ``observadores'' serão notificados através de uma única mensagem em \emph{multicast}
de quando os dados em uma fonte sofrerem alterações, dessa forma fica mais fácil a criação de relacionamentos um para muitos (1:N).~\cite{Domingues2016}. 

\subsubsection{ZigBee}
~\label{sec:zigbee}
ZigBee é um dos mais populares padrões de redes \emph{mesh} sem fio. O protocolo é aberto e baseado em pacotes, projetado para ser de fácil uso,
baixo consumo e seguro~\cite{tomar2011}.
O modelo ZigBee é principalmente definido em três camadas.
\begin{itemize}
    \item \emph{Application Support Sublayer}: Responsável por vincular \emph{endpoints}, transmistir mensagens entre os dispositivos e o gerenciamento de grupos.
    \item \emph{ZigBee Device Object}: Responsável pelo gerenciamento de dispositivos; definir o modo de operação do dispositivo; descobrir novos dispositivos e quais serviços de aplicação ele provê; lidar com os pedidos de vinculamento de outros dispositivos.
    \item \emph{Application Framework}: Abriga as aplicações dos dispositivos. 
\end{itemize}
~\cite{Domingues2016}

\subsection{Tecnologias utilizadas no desenvolvimento deste trabalho}
\subsubsection{Tasmota}
~\label{sec:tasmota}
Tasmota é um software de código aberto que atua como firmware para SoC’s (System-on-a-Chip) ESP32 
da Espressif e possibilita controle e comunicação personalizado com alguns dispositivos comerciais que são 
baseados nesse SoC, como interruptores inteligentes, lâmpadas inteligentes ou controles infravermelho
inteligentes.
A comunicação com o firmware é realizada através do protocolo MQTT, permitindo integrar o hardware comercial
a qualquer sistema com suporte ao protocolo~\cite{tasmota-1}.

\subsubsection{MQTT}
~\label{sec:mqtt}
MQTT ou Message Queuing Telemetry Transport~\cite{ISO20922} é um protocolo de transporte para mensagens no formato \emph{publicar/assinar},
ele é simples, leve, aberto e projetado para ser fácil de implementar, seu foco de uso são ambientes onde recursos são geridos com cuidado,
como comunicação \emph{Machine to Machine} (M2M) e Internet das Coisas~\cite{MQTT5}. O foco principal do projeto desse protocolo é minimizar
o consumo de banda e recursos do dispositivo, sendo assim, ele é capaz de transmitir dados com uma banda baixa e conexões instáveis.
O protocolo funciona sobre TCP/IP ou outros protocolos de rede que possam prover as mesmas capacidades do TCP/IP como conexões bi-direcionais.
O MQTT oferece três qualidades de serviços (\emph{Quality of Service}, ou QoS) na entrega de mensagens.
\begin{itemize}
    \item QoS 0: ``No máximo uma vez'', mensagens são entregues no máximo uma vez.
    \item QoS 1: ``Pelo menos uma vez'', mensagens são garantidas de serem entregues pelo menos uma vez, pode ocorrer duplicatas.
    \item QoS 2: ``Exatamente uma vez'', mensagens são entregues exatamente uma vez.
\end{itemize}
Uma estrutura que suporte o protocolo MQTT deve ter pelo menos três componentes,
\begin{enumerate}
    \item \emph{Publisher} ou Produtor, um cliente MQTT
    \item \emph{Broker}, o servidor MQTT que gerencia o recebimento e entrega das mensagens
    \item \emph{Subscriber} ou Consumidor, um cliente MQTT
\end{enumerate}
No padrão \emph{publicar/assinar}, o produtor cria uma mensagem e envia para o servidor, consumidores então leem essa mensagem.
No MQTT um produtor publica mensagens nos tópicos do \emph{broker}, consumidores interessados em receber essas mensagens se
inscreverão nesse tópico e o \emph{broker} se encarregará de entregá-las~\cite{Mishra2020}.



