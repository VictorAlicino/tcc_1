\chapter{Internet das Coisas}

Enquanto a internet foi desenvolvida com dados criados por pessoas, 
a Internet das Coisas é sobre os dados criados por objetos \cite{Nord2019}. 
Madakam, Ramaswamy e Tripathi definem Internet das Coisas como uma rede de objetos inteligente 
aberta que tem a capacidade de se auto-organizar, compartilhar informações, dados e 
recursos \cite{Madakam2015}.

Internet das Coisas é muitas vezes referido pela sigla \verb'IoT' que vem do 
seu nome em inglês \textit{Internet of Things}. No nome dessa tecnologia temos as 
duas partes que desempenham os principais papéis: \emph{Internet} e \emph{Coisas}.
\emph{Internet} aqui, se refere a mesma internet usadas por bilhões de pessoas ao 
redor do mundo, já \emph{Coisas}, se refere aos dispositivos com capacidades computacionais 
atrelados a sensores ou atuadores conectados à rede, o que permite a eles trocarem e 
consumirem informações.

Em uma rede \verb'IoT', dispositivos (muitas vezes chamados de \emph{inteligentes}) tem a habilidade
de se comunicarem para monitorar o ambiente em que estão, ou alterar este ambiente. Essas ações 
podem ser configuradas previamente pelo usuário ou serem definidas na hora com a possibilidade de 
ser acionadas fora da rede local onde esses dispositivos estão conectados através da internet.
Como por exemplo, verificar a temperatura de uma sala sem ter chego nela e solicitar ao sistema que 
acione as unidades AVAC para que a sala esteja na temperatura desejada ao chegar.

Apesar de ocorrer leves diferenças nos modelos de arquitetura, 
geralmente um sistema IoT é formado por três camadas, sendo elas:

\begin{itemize}
    \item Camada Física: responsável por perceber o ambiente físico com sensores, 
    atuadores ou qualquer tipo de interface para perceber e modificar o ambiente físico;

    \item Camada de Rede: camada de transmissão dos dados através das inúmeras formas de conexões 
    disponíveis como redes sem fio, redes móveis e a internet, a fim de fornecer os dados da 
    Camada Física para a Camada de Aplicação;

    \item Camada de Aplicação, que oferece os serviços chamados \emph{inteligentes} para os usuários 
    finais.
\end{itemize}
\cite{Yan2014}.

\begin{figure}[h!]
    \conteudoFigura
    [\citeonline{statista-iot-connected-devices}]         % Fonte   
    {0.3}                               % scale: 0.01 - 1
    {statistic_id1183457.jpg}                 % filename. A figura deve estar na pasta Imagens
    {Números de dispositivos IoT conectados no mundo de 2019 à 2023}     % Texto
    {fig:statistic_01}                      % identificador da figura
\end{figure}

Segundo a empresa especializada em coleta de dados, Statista, 
a quantidade de dispositivos \verb'IoT' conectados no mundo passa de 15 bilhões em 
2023 com projeção para quase o dobro deste número até o fim da década 
\cite{statista-iot-connected-devices} como mostrado na figura \ref{fig:statistic_01}. 

Isso se deve muito ao fato de que sensores e microcontroladores ficaram mais baratos, o 
preço médio dos sensores caiu 200\% entre 2004 e 2018 \cite{microsoft-1} o que significa
mais produtos finais voltado para \verb'IoT' já que ficou possível embarcar sensores em itens 
do dia a dia com custos acessíveis.

\begin{figure}[h!]
    \conteudoFigura
    [\citeonline{Minoli2017}]         % Fonte   
    {0.8}                               % scale: 0.01 - 1
    {Minoli2017.jpg}                 % filename. A figura deve estar na pasta Imagens
    {Convergência da tecnologia de sistemas inteligentes para edifícios nos últimos anos}     % Texto
    {fig:acs-iot-evolution}                      % identificador da figura
\end{figure}

Alguns anos atrás, mesmo em edifícios com BACS, não era incomum encontrar diferentes protocolos, 
redes e cabos, o que gerava ineficiência tanto na implantação como nos sistemas de gerenciamento. 
Dessa forma algumas soluções migraram para o uso de cabo e conjunto de protocolos comuns entre as 
aplicações, como o cabo de par trançado categoria 6 e o protocolo TCP/IP. 
Com a convergência para protocolos como TCP/IP facilitou a entrada do \verb'IoT' em BACS, 
como mostrado na figura 
\ref{fig:acs-iot-evolution} \cite{Minoli2017}.

Já é possível encontrar equipamentos preparados para uso em BACS com suporte a protocolos \verb'IoT',
como o PFC200 da WAGO que suporta o protocolo MQTT (citado na seção \ref{sec:mqtt}).

\begin{figure}[h!]
    \conteudoFigura
    [\citeonline{wago-1}]         % Fonte   
    {0.10}                               % scale: 0.01 - 1
    {wago-pfc200.png}                 % filename. A figura deve estar na pasta Imagens
    {Controlador PFC200}     % Texto
    {fig:pfc200}                      % identificador da figura
\end{figure}