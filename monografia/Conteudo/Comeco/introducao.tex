\chapter{Introdução} 

% Contextualização do problema
As cidades inteligentes visam melhorar a qualidade de vida dos cidadãos e promover um desenvolvimento econômico e social sustentável por meio do 
uso de tecnologias da informação e comunicação (TIC)~\cite{cetic}. 
Nesse contexto, os edifícios inteligentes desempenham um papel fundamental. Indivíduos passam grande parte da sua vida dentro de edifícios, crescem, 
estudam e se desenvolvem neles~\cite{art1}~\cite{noauthor_how_nodate}. Dada sua importância no cotidiano urbano, a modernização dos edifícios 
torna-se essencial para a implementação do conceito de cidades inteligentes. Assim, a integração de novas tecnologias pode otimizar 
a experiência dos ocupantes e contribuir para a sustentabilidade dessas construções.

Para que um edifício seja classificado como inteligente, é necessária a incorporação de dispositivos capazes de monitorar e controlar o ambiente.
Esses dispositivos incluem sensores, responsáveis por captar os dados do ambiente em que se encontram; atuadores que realizam as ações
físicas baseadas nas entradas que recebem; e controladores que coordenam a comunicação entre sensores e atuadores para garantir uma automação
eficiente~\cite{Morvaj2011}. A implementação desses componentes possibilita desde o gerenciamento de iluminação e climatização até sistemas avançados
de segurança e eficiência energética.

Nos últimos anos, dispositivos com essas características têm se tornado cada vez mais populares no meio da Internet das Coisas (do inglês, IoT).
Com a significativa redução de custo dos sensores nas últimas duas décadas~\cite{microsoft1}, dispositivos IoT destinados 
a Casas Inteligentes ganharam ampla aceitação no mercado, devido ao seu custo acessível e facilidade na instalação, com isso, estima-se que ao final 
da década de 2020, dos mais de 30 bilhões de dispositivos IoT conectados no mundo, 
60\% serão de consumidores finais\cite{statista-iot-connected-devices}.

Diante da crescente adoção de dispositivos IoT em Casas Inteligentes, surge a questão de por que esses dispositivos ainda não são amplamente utilizados
em edifícios coletivos, como escritórios e coworking, para torná-los inteligentes. Os sistemas de automação tradicionais, projetados para grandes edifícios,
costumam ser caros e complexos, enquanto os dispositivos IoT voltados ao consumidor final são acessíveis e fáceis de instalar. No entanto, esses dispositivos
foram concebidos para um uso individual ou residencial, assumindo que os ocupantes do espaço são previamente conhecidos. Em edifícios coletivos, essa 
abordagem se torna inviável, pois a rotatividade de usuários exige um modelo flexível e adaptável de controle e segurança. Assim, há um desafio técnico e 
econômico a ser superado para que dispositivos IoT de Casas Inteligentes possam ser utilizados nesses ambientes de forma segura e eficiente.

% Problema de Pesquisa
Apesar dessa ampla adoção no mercado residencial, a aplicação desses dispositivos em edifícios coletivos ainda enfrenta desafios específicos.
Considerando que dispositivos IoT de Casas Inteligentes são mais acessíveis e fáceis de implementar que sistemas tradicionais de automação predial,
mas não são viáveis em edifícios coletivos devido à necessidade de múltiplos usuários compartilharem o mesmo ambiente, como é possível adaptá-los
para o uso em edifícios coletivos, garantindo controle adequado para múltiplos ocupantes?

Dado que dispositivos IoT de Casas Inteligentes são mais acessíveis e fáceis de implementar do que os sistemas tradicionais de automação predial, 
como é possível adaptá-los para o uso em edifícios coletivos, garantindo segurança e controle adequado para múltiplos ocupantes?

% Justificativa

% Objetivo Geral
\section{Objetivos}
\subsection{Objetivo Geral}
Este trabalho visa desenvolver um sistema básico que demonstre a viabilidade de múltiplos usuários operarem simultaneamente 
um sistema de edifício inteligente.

% Objetivos Específicos	
\subsection{Objetivos Específicos}

Para alcançar o objetivo do trabalho, é proposto uma aplicação simplificada que permita entender a viabilidade do uso de um 
sistema de edifício inteligente por diversos usuários, focando no controle das unidades de climatização.

Dentre os principais objetivos específicos destacam-se:

\begin{itemize}
    \item Realizar uma pesquisa bibliográfica sobre edifícios inteligentes e suas demandas;
    \item Estudar sobre as aplicações da internet das coisas em um edifício inteligente;
    \item Definir requisitos para um sistema de gerenciamento de um edifício inteligente;
    \item Propor e implementar um sistema de gerenciamento para edifícios inteligentes;
\end{itemize}

% Estrutura do trabalho (explicando brevemente o que será abordado em cada seção)
\section{Estrutura do Trabalho}
Este trabalho está organizado com a seguinte estrutura:

\textbf{Capítulo 2}: Revisão Bibliográfica, onde são apresentados\dots

\textbf{Capítulo 3}: Arquitetura do Sistema\dots.

\textbf{Capítulo 4}: Dispositivo IoT\dots

\textbf{Capítulo 5}: Servidor Local\dots

\textbf{Capítulo 6}: Servidor Remoto\dots

\textbf{Capítulo 7}: Interface de Usuário\dots

\textbf{Capítulo 8}: Conclusão\dots
