\chapter{Sistema Opus}

Buscando o objetivo do trabalho, foi proposto um sistema denominado de \textbf{Opus}, dividido em três partes, 
\emph{Opus} (Servidor Local), \emph{Maestro} (Servidor Remoto) e \emph{Conductor} (Interface de Usuário), 
ele tem o objetivo principal de permitir que múltiplos usuários possam controlar dispositivos IoT.

O sistema \textbf{Opus} foi desenvolvido pensando em edifícios de uso coletivo, como escritórios, coworking e
salas de aula, onde a automação pode ser utilizada para facilitar a vida dos usuários, como controlar a iluminação e
climatização. 

Inicialmente foi planejado um sistema apenas com Servidor e Interface de Usuário, porém, visando a autonomia de edifícios e
privacidade dos dados, o Servidor foi divido em dois, Servidor Local (que manteve o nome \emph{Opus}) e Servidor Remoto (que recebeu o nome de 
\emph{Maestro}), onde o Servidor Local é responsável por todo o gerenciamento de dispositivos e permissões de usuários, 
enquanto o Servidor Remoto é responsável pela autenticação de usuários e sincronização de informações entre os Servidores Locais,
assim como ser uma ponte entre a Interface de Usuário e os Servidores Locais através da internet, permitindo o controle remoto
dos dispositivos em qualquer lugar do mundo.

A arquitetura do sistema é exemplificada na figura~\ref{fig:overall}.

\begin{figure}[h!]
    \conteudoFigura
    [Elaborado pelo Autor (2025)]
    {0.2}
    {overall.png}
    {Arquitetura do Sistema}
    {fig:overall}
\end{figure}

A comunicação entre as partes do sistema é realizada através do protocolo MQTT. 
Este protocolo foi escolhido por ser leve e fácil de implementar utilizando bibliotecas disponíveis para Python,
a linguagem de programação escolhida para o desenvolvimento do sistema. Dada a natureza ``Publicar e Assinar'' do MQTT,
ele não é ideal para aplicações do tipo ``Requisição e Resposta'', para contornar este detalhe, foi implementado um padrão na comunicação entre
\emph{Opus} e \emph{Maestro}.

Todas as mensagens devem ser do tipo JSON, contendo sempre um campo ``callback'' que indicará ao destinatário da mensagem o tópico MQTT
onde a resposta deve ser enviada ao remetente. O remetente deve se inscrever no tópico ``callback'' para receber a resposta, como ilustrado
na figura~\ref{fig:mqtt_comm_1}, onde ``\lstinline{data}'' é o dado que será mandado do \emph{Maestro} para o \emph{Opus}, note que, além do
campo do conteúdo da mensagem (``\lstinline{device_id}''), também é enviado um campo ``\lstinline{callback}'' que indica o tópico MQTT onde a resposta
deve ser enviada, esse campo possui um identificador aleatório, garantindo que várias respostas possam ser recebidas ao mesmo tempo sem conflitos.
\begin{figure}[h!]
    \conteudoFigura
    [Elaborado pelo Autor (2025)]
    {0.3}
    {mqtt_comm_example.png}
    {Exemplo de Comunicação entre \emph{Opus} e \emph{Maestro}}
    {fig:mqtt_comm_1}
\end{figure}

