\section{\textbf{Servidor Remoto}}
\label{chap:maestro}
Em um ambiente de edifícios inteligentes, a gestão dos usuários deve ser executada com cautela, pois certos dispositivos não podem
ser acessados por qualquer usuário, assim como os usuários devem ser rastreáveis para evitar que as funcionalidades que o sistema adiciona
possam se tornar prejudiciais para o edifício e sua gestão, como mal uso de dispositivos, ou até mesmo ações maliciosas. Com estas limitações
em mente foi montada a dinâmica de usuários do sistema Opus, onde os servidores Opus são responsáveis por gerenciar permissões
de acesso a dispositivos e uma entidade centralizada é responsável por autenticar e gerenciar usuários. Com a necessidade dessa entidade centralizada
foi criado o Maestro, um servidor remoto responsável pela autenticação e gestão de usuários, bem como a intermediação entre o aplicativo móvel e 
os servidores locais.

Sendo assim, as duas principais funções do Maestro são:
\begin{itemize}
    \item Autenticação e gerenciamento de usuários;
    \item Intermediação entre o aplicativo móvel e os servidores locais.
\end{itemize}
Essas funções são realizadas por 2 módulos principais que controlam duas API (Application Programming Interface) distintas, a API em 
REST (Representational State Transfer), voltada a comunicação com o usuário e a API em MQTT voltada a comunicação com os servidores locais.

\begin{figure}[h!]
    \conteudoFigura
    [Elaborado pelo Autor (2025)]
    {0.35}
    {maestro.png}
    {Módulos do Maestro}
    {fig:maestro1}
\end{figure}

\subsection{Autenticação e Cadastro de Usuários}
\label{sec:maestro-auth}

A autenticidade da identidade do usuário foi uma preocupação desde o início do desenvolvimento do sistema \emph{Opus}, tendo em vista que criar um mecanismo
de autenticação próprio poderia ser uma tarefa muito complexa e o resultado poderia não garantir a autenticidade esperada, o próximo passo foi buscar
algum serviço de autenticação que pudesse garantir a identidade dos usuários, e nisso foi escolhido o sistema de autenticação do Google, que já é amplamente
utilizado e confiável. Além disso, o sistema de autenticação do Google permite que o aplicativo móvel do Opus possa ser utilizado por qualquer pessoa que
possua uma conta Google, o que facilita a adoção do sistema.

\begin{figure}[h!]
    \conteudoFigura
    [Elaborado pelo Autor (2025)]
    {0.3}
    {google-sign-payload.png}
    {Informações do Google Auth}
    {fig:google-auth}
\end{figure}

A autenticação do Google entrega seis informações sobre um usuário, como ilustrado na figura~\ref{fig:google-auth},
sendo as três mais importantes o \emph{email}, o \emph{nome} e o \emph{ID} do usuário na Google, o \emph{Maestro} utiliza essas informações para
criar um usuário no banco de dados, acrecidos de um novo \emph{user\_id} que segue o padrão utilizado em outras tabelas do sistema,
a tabela de usuário do \emph{Maestro} é composta pelas seguintes informações:

\begin{itemize}
    \item \textbf{Maestro User}  
      \begin{itemize}
          \item \texttt{user\_id}: Identificador único do usuario;
          \item \texttt{email}: E-mail do usuário;
          \item \texttt{name}: Primeiro e último nome do usuário;
          \item \texttt{given\_name}: Primeiro nome do usuário;
          \item \texttt{family\_name}: Último nome do usuário;
          \item \texttt{google\_sub}: Identificador único do usuário no Google.
          \item \texttt{picture\_url}: URL da foto de perfil do usuário.
          \item \texttt{picture}: Foto de perfil do usuário em bytea.
      \end{itemize}
\end{itemize}

Transferindo a responsabilidade de garantir a autenticidade de um usuário com uma conta para o Google, o \emph{Maestro} pode garantir que os 
utilizadores do sistema são de alguma forma rastreáveis caso seja necessário. 

Quando é necessário associar um usuário do \emph{Maestro} a um servidor Opus, como o caso de marcar um usuário como administrador de um servidor, o \emph{Maestro} passará 
os dados de nome, email e \emph{user\_id} para o servidor Opus, que irá sempre depender do \emph{Maestro} para garantir a autenticidade dos usuários.
\subsection{Intermediação entre a Interface de Usuário e os Servidores Locais}
\label{sec:maestro-intermediacao}

O Maestro atua como intermediário entre a interface de usuário e os servidores locais Opus,
fornecendo dois meios de comunicação distintos para atender às diferentes necessidades do sistema:
\begin{itemize}
    \item API REST,
    \item API MQTT;
\end{itemize}

\subsubsection{API REST para Interface de Usuário}
A API REST do Maestro expõe os controles do Servidor Remoto e comunicação com os Servidores Locais, divididos em quatro categorias:
\begin{itemize}
    \item Autenticações, com o endpoint \lstinline{/auth}:
    \item Usuários, com os endpoints \lstinline{/users};
    \item Servidores Locais, com os endpoints \lstinline{/opus_server};
    \item Dispositivos, com os endpoints \lstinline{/device};
\end{itemize}

\textbf{Autenticação}, são endpoints que lidam com a autenticação e cadastro de usuários, endpoints específicos para usuários
só podem ser acessados por usuários autenticados.
\textbf{Usuários}, são endpoints que lidam com a gestão de usuários, como listar todos, listar um usuário específico, listar os Servidores Locais
que este usuário tem acesso e também listar todos os dispositivos que este usuário tem acesso.
\textbf{Servidores Locais}, são endpoints que lidam com a gestão de Servidores Locais, como designar um usuário como administrador ou usuário de um Servidor Local.
\textbf{Dispositivos}, por último, os endpoints de dispositivos permitem se comunicar diretamente com um dispositivo em um Servidor Local, 
como ligar ou desligar um dispositivo.

\subsubsection{API MQTT para Servidores Locais}
A API MQTT do Maestro é responsável por intermediar a comunicação entre a Interface de Usuário e os Servidores Locais, divididos em apenas dois módulos de 
comunicação: Dispositivos e Usuários.

O módulo de \textbf{Dispositivos} acessa os tópicos MQTT de um Servidor Local para controlar e receber informações de dispositivos, ,
omo ligar ou desligar, ou saber o atual estado de um dispositivo. 
Já o módulo de \textbf{Usuários} é responsável por comandos como recuperar todos os dispositivos que um usuário tem acesso em um Servidor Local e 
designar um usuário a um servidor, seja qual for seu nível de acesso.
\subsection{Banco de Dados}
\label{sec:maestro-banco-dados}

O banco de dados do \emph{Maestro} é composto por apenas 3 tabelas, sendo elas: \emph{Maestro User}, \emph{Opus Server} e \emph{Roles}.
A tabela \emph{Maestro User} é responsável por armazenar os dados dos usuários autenticados, a tabela \emph{Opus Server} é responsável por armazenar
os dados dos servidores locais e a tabela \emph{Roles} é responsável por armazenar os níveis de acesso dos usuários a cada servidor, sendo uma 
tabela de relacionamento entre \emph{Maestro User} e \emph{Opus Server}.

O diagrama do banco de dados é ilustrado na figura~\ref{fig:maestro-db}.

\begin{figure}[h!]
    \conteudoFigura
    [Elaborado pelo Autor (2025)]
    {0.4}
    {erd_maestro.png}
    {Diagrama do Banco de Dados do Maestro}
    {fig:maestro-db}
\end{figure}