\section{Servidor Remoto}
\label{chap:maestro}
Em um ambiente de edifícios inteligentes, a gestão dos usuários deve ser executada com cautela, pois certos dispositivos não podem
ser acessados por qualquer usuário, assim como os usuários devem ser rastreáveis para evitar que as funcionalidades que o sistema adiciona
possam se tornar prejudiciais para o edifício e sua gestão, como mal uso de dispositivos, ou até mesmo ações maliciosas. Com estas limitações
em mente foi montada a dinâmica de usuários do sistema Opus, onde os servidores Opus [\ref{opus}] são responsáveis por gerenciar permissões
de acesso a dispositivos e uma entidade centralizada é responsável por autenticar e gerenciar usuários. Com a necessidade dessa entidade centralizada
foi criado o Maestro, um servidor remoto responsável pela autenticação e gestão de usuários, bem como a intermediação entre o aplicativo móvel e 
os servidores locais.

Sendo assim, as duas principais funções do Maestro são:
\begin{itemize}
    \item Autenticação e gerenciamento de usuários;
    \item Intermediação entre o aplicativo móvel e os servidores locais.
\end{itemize}
Essas funções são realizadas através de 2 módulos principais que controlam duas API (Application Programming Interface) distintas, a API em 
REST (Representational State Transfer), voltada a comunicação com o usuário e a API em MQTT voltada a comunicação com os servidores locais.

\begin{figure}[h!]
    \conteudoFigura
    [Elaborado pelo Autor (2025)]
    {0.35}
    {maestro.png}
    {Módulos do Maestro}
    {fig:maestro1}
\end{figure}

\subsection{API REST}

Além de gerenciar a autenticação e cadastro de usuários, a API REST também é responsável por agir como intermediador entre os Servidores Locais \emph{Opus} e
a Interface de Usuário, ela faz isso utilizando a API MQTT, que envia comandos utilizando esse protocolo que é o padrão de comunicação entre servidores no sistema
\emph{Opus}.