\chapter{Servidor Remoto}
\label{chap:maestro}
Em um ambiente de edifícios inteligentes, a gestão dos usuários deve ser executada com cautela, pois certos dispositivos não podem
ser acessados por qualquer usuário, assim como os usuários devem ser rastreáveis para evitar que as funcionalidades que o sistema adiciona
possam se tornar prejudiciais para o edifício e sua gestão, como mal uso de dispositivos, ou até mesmo ações maliciosas. Com estas limitações
em mente foi montada a dinâmica de usuários do sistema Opus, onde os servidores Opus [\ref{opus}] são responsáveis por gerenciar permissões
de acesso a dispositivos e uma entidade centralizada é responsável por autenticar e gerenciar usuários. Com a necessidade dessa entidade centralizada
foi criado o Maestro, um servidor remoto responsável pela autenticação e gestão de usuários, bem como a intermediação entre o aplicativo móvel e 
os servidores locais.

Sendo assim, as duas principais funções do Maestro são:
\begin{itemize}
    \item Autenticação e gerenciamento de usuários;
    \item Intermediação entre o aplicativo móvel e os servidores locais.
\end{itemize}

