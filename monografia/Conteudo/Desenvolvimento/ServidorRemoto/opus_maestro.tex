\subsection{Intermediação entre a Interface de Usuário e os Servidores Locais}
\label{sec:maestro-intermediacao}

O Maestro atua como intermediário entre a interface de usuário e os servidores locais Opus,
fornecendo dois meios de comunicação distintos para atender às diferentes necessidades do sistema:
\begin{itemize}
    \item API REST,
    \item API MQTT;
\end{itemize}

\subsubsection{API REST para Interface de Usuário}
A API REST do Maestro expõe os controles do Servidor Remoto e comunicação com os Servidores Locais, divididos em quatro categorias:
\begin{itemize}
    \item Autenticações, com o endpoint \lstinline{/auth}:
    \item Usuários, com os endpoints \lstinline{/users};
    \item Servidores Locais, com os endpoints \lstinline{/opus_server};
    \item Dispositivos, com os endpoints \lstinline{/device};
\end{itemize}

\textbf{Autenticação}, são endpoints que lidam com a autenticação e cadastro de usuários, endpoints específicos para usuários
só podem ser acessados por usuários autenticados.
\textbf{Usuários}, são endpoints que lidam com a gestão de usuários, como listar todos, listar um usuário específico, listar os Servidores Locais
que este usuário tem acesso e também listar todos os dispositivos que este usuário tem acesso.
\textbf{Servidores Locais}, são endpoints que lidam com a gestão de Servidores Locais, como designar um usuário como administrador ou usuário de um Servidor Local.
\textbf{Dispositivos}, por último, os endpoints de dispositivos permitem se comunicar diretamente com um dispositivo em um Servidor Local, 
como ligar ou desligar um dispositivo.

\subsubsection{API MQTT para Servidores Locais}
A API MQTT do Maestro é responsável por intermediar a comunicação entre a Interface de Usuário e os Servidores Locais, divididos em apenas dois módulos de 
comunicação: Dispositivos e Usuários.

O módulo de \textbf{Dispositivos} acessa os tópicos MQTT de um Servidor Local para controlar e receber informações de dispositivos, ,
omo ligar ou desligar, ou saber o atual estado de um dispositivo. 
Já o módulo de \textbf{Usuários} é responsável por comandos como recuperar todos os dispositivos que um usuário tem acesso em um Servidor Local e 
designar um usuário a um servidor, seja qual for seu nível de acesso.