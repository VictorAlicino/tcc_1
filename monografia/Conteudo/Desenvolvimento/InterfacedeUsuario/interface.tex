\section{\textbf{Interface de Usuário}}

A Interface de Usuário do sistema \emph{Opus}, denominada de \emph{Conductor}, ela é responsável por acessar os endpoints 
expostos pelo Servidor Remoto \emph{Maestro} e atua como um invólucro para o usuário final interagir com o sistema.

O \emph{Conductor} é um aplicativo para dispositivos móveis, desenvolvido utilizando o \emph{framework} React,
que possui as seguintes funções:
\begin{itemize}
    \item Capturar as informações do usuário para autenticação;
    \item Exibir os dispositivos disponíveis para o usuário;
    \item Permitir o controle dos dispositivos.
\end{itemize}

Em resumo, o \emph{Conductor} é um aplicativo simples, que permite ao usuário interagir com o sistema \emph{Opus}, se comunicando 
com o Servidor Remoto \emph{Maestro} por requisições HTTP que se comunica com os Servidores Locais \emph{Opus} através do protocolo MQTT
como ilustrado na figura~\ref{fig:conductor-overall}.

\begin{figure}[h!]
    \conteudoFigura
    [Elaborado pelo Autor (2025)]
    {0.25}
    {conductor_overall.png}
    {Comunicação no \emph{Conductor}}
    {fig:conductor-overall}
\end{figure}

\subsection{Autenticação}
\label{subsec:conductor-auth}

O \emph{Conductor} é uma parte fundamental da autenticação no sistema \emph{Opus}, uma vez que é muito mais fácil de implementar uma autenticação com Google
em um aplicativo móvel e muito mais seguro para o usuário final.

Foi usada uma biblioteca chamada \emph{react-native-google-signin} que facilita a autenticação com o Google, ela é uma \emph{wrapper} da biblioteca
\emph{Google Sign-In} para Android e iOS, que permite ao usuário autenticar com o Google em ambos os sistemas operacionais. Quando o usuário
clica no botão de ``Continuar com Google'', uma tela de seleção de contas do Google é aberta, onde o usuário pode escolher a conta que deseja,
como ilustrado na figura~\ref{fig:conductor-auth}.

\begin{figure}[ht]
    \conteudoFigura
    [Elaborado pelo Autor (2025)]
    {0.15}
    {google-conductor-auth.png}
    {Janela do Google de Autenticação}
    {fig:conductor-auth}
\end{figure}

O fluxo de autenticação é simples, uma vez que o usuário escolhe a conta, o \emph{Conductor} recebe as informações da conta Google do usuário, enviadas 
para o \emph{Maestro} por meio de uma requisição HTTP, que por sua vez, autentica o usuário e retorna um \emph{token} de autenticação JWT, 
armazenado no \emph{Conductor} e utilizado em todas as requisições subsequentes ao \emph{Maestro}.

\begin{figure}[ht]
    \conteudoFigura
    [Elaborado pelo Autor (2025)]
    {0.23}
    {conductor_auth.png}
    {Fluxo de Autenticação no \emph{Conductor}}
    {fig:conductor-auth-flow}
\end{figure}

\subsection{Exibir Dispositivos}

\begin{figure}[ht]
    \conteudoFigura
    [Elaborado pelo Autor (2025)]
    {0.15}
    {conductor-home-building-rooms.png}
    {Tela Home, Tela de Seleção de Edifícios e Tela de Seleção de Dispositivos}
    {fig:conductor-home}
\end{figure}

Ao ser autenticado, o usuário será recebido por uma tela contendo as salas de um edifício registrado em um dos servidores \emph{Opus}
que ele tem permissão de acesso, como ilustrado na figura~\ref{fig:conductor-home}.

Também é possível alterar o edifício selecionado, clicando no nome do edifício no canto superior da tela, onde será exibida uma lista de edifícios
autorizados para o usuário, ilustrado na figura~\ref{fig:conductor-home}.

Ao selecionar uma sala, o usuário será redirecionado para uma tela contendo os dispositivos disponíveis na sala, 
como também ilustrado na figura~\ref{fig:conductor-home}.

\subsection{Controlar Dispositivos}
\label{sec:conductor-control}

Por fim, ao selecionar um dispositivo, o usuário será redirecionado para uma tela de controle do dispositivo, onde poderá
interagir com os comandos específicos daquele tipo de dispositivo, na figura~\ref{fig:conductor-control} é possível ver a tela de 
controle de uma unidade AVAC (Aquecimento, Ventilação e Ar Condicionado).

\begin{figure}[ht]
    \conteudoFigura
    [Elaborado pelo Autor (2025)]
    {0.2}
    {conductor_hvac.jpg}
    {Tela de Controle de um Dispositivo AVAC}
    {fig:conductor-control}
\end{figure}

