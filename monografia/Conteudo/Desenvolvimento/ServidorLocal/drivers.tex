\subsection{Drivers}

Sem uma padronização da comunicação entre dispositivos IoT, cada fabricante pode adotar protocolos distintos e estruturas variadas de mensagens.
Diante deste cenário, foi adotado o conceito de \emph{Driver}s, criado para abstrair e padronizar a comunicação entre dispositivos e o \emph{Opus}.

A função de um \emph{Driver} é atuar como um tradutor entre o \emph{Opus} e um tipo específico de dispositivo, garantido que comandos e dados
sejam corretamente interpretados e processados. Isso garante a integração de uma ampla gama de dispositivos sem a necessidade de modificar os 
módulos do sistema, como resultado o sistema se torna modular e escalável.

Em casos onde o dispositivo se comunica via um protocolo não suportado nativamente pelo Python, como dispositivos que requerem uma camada
física diferente das utilizadas pelo protocolo TCP/IP, como Serial, ou até mesmo que utilizam TCP/IP, porém não HTTP, como o MQTT,
é possível que \emph{Driver}s acessem as \emph{Interfaces}[\ref{opus-sec:interfaces}] para realizar a comunicação com o dispositivo.

