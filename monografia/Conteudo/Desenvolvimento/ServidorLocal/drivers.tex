\subsection{Drivers}

% Por que os drivers são necessários no Opus?
% Quais problemas o conceito de drivers resolve?
% Quais são as responsabilidades de um driver?
% Como os drivers funcionam dentro do Opus?
% Os drivers se comunicam diretamente com os dispositivos ou há camadas intermediárias?
% Os drivers seguem algum padrão de implementação dentro do Opus?
% Como um novo driver pode ser desenvolvido e integrado ao Opus?
% Quais são os desafios ao criar um driver para um novo dispositivo IoT?
% Quais drivers já foram implementados no Opus?
% Há algum driver planejado, mas que não será finalizado para o TCC? Por quê?

Sem uma padronização da comunicação entre dispositivos IoT, o Opus não tem como prever todos as particularidades 
de comunicação de cada dispositivo, com isto em mente foi criado o conceito de \emph{drivers} para abstrair as comunicações.

A função de um driver é essencialmente agir como um tradutor entre o Opus e um tipo de dispositivo, desta forma é possível
abrangir uma gama maior de dispositivos sem a necessidade de modificar o código de outros módulos do sistema.





Os \emph{drivers} são responsáveis por traduzir os comandos do sistema para os dispositivos, de forma que cada dispositivo no \emph{Opus} está
associado a um \emph{driver} específico, desta forma, quando o usuário pedir que um determinado dispositivo seja ligado, por exemplo,
esse comando chamará o método correspondente da classe do dispositivo que utilizará seu \emph{driver} para enviar o comando para o hardware.

Usando o Tasmota como exemplo, um dos comandos que é possível usar, é o de ligar um AVAC, para isso é necessário enviar uma mensagem
MQTT em uma estrutura de JSON para o tópico: ``\lstinline{cmnd/TOPICO MQTT DO DISPOSITIVO/IRHVAC}'', com o seguinte conteúdo:
\begin{lstlisting}[language=Python]
    {
    "Vendor": O fabricante do AVAC,
    "Power": AVAC ligado ou desligado,
    "Mode":  Modo do AVAC, como frio, calor, ventilação, etc,
    "FanSpeed": Velocidade do ventilador,
    "Temp": Temperatura desejada
    }
\end{lstlisting}

A função do driver é abstrair detalhes como estes e expor para aplicação apenas os comandos que ela espera ter, como ligar, desligar, etc.
Para ligar um AVAC Tasmota no \emph{Opus} utilizando este driver, a aplicação apenas chama o método ``\lstinline{on()}'' do objeto. O
método ``\lstinline{on()}'' está ilustrado na figura \ref{fig:driver1}.
\begin{figure}[h!]
    \conteudoFigura
    [Elaborado pelo Autor (2024)]
    {0.35}
    {driver1.png}
    {Método \lstinline{on()} de dispositivos AVAC com o \emph{driver} Tasmota}
    {fig:driver1}
\end{figure}

