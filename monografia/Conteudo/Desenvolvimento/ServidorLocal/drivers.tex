\subsection{Drivers}

Sem uma padronização da comunicação entre dispositivos IoT, cada fabricante pode adotar protocolos distintos e estruturas variadas de 
mensagens. Diante deste cenário, foi adotado o conceito de \emph{Driver}s, criado para abstrair e padronizar a comunicação entre 
dispositivos e o \emph{Opus}.

Assim como as \emph{Interface}s[\ref{opus-sec:interfaces}], os \emph{Driver}s são módulos escritos em Python carregados no momento da inicialização do \emph{Opus},
sua função é atuar como um tradutor entre o \emph{Opus} e um tipo específico de dispositivo, garantido que 
comandos e dados sejam corretamente interpretados e processados. Isso garante a integração de uma ampla gama de dispositivos 
sem a necessidade de modificar os  módulos do sistema, como resultado o sistema se torna modular e escalável.

Em casos onde o dispositivo se comunica via um protocolo não suportado nativamente pelo Python, como dispositivos que 
requerem uma camada física diferente das utilizadas pelo protocolo TCP/IP, como Serial, ou até mesmo que utilizam TCP/IP, 
porém não HTTP, como o MQTT, é possível que \emph{Driver}s acessem as \emph{Interfaces} para realizar a comunicação com o dispositivo.

Para ser considerado um \emph{Driver}, o módulo deve ter implementado uma função \lstinline{start()} sem retornos, essa função será 
chamada no momento em que o \emph{Driver} for carregado, também deve ter um dicionário global de \emph{Interface}s com a seguinte assinatura
\lstinline{interfaces: dict[str, Any]}, em caso do \emph{Driver} necessitar de uma \emph{Interface}, o nome dela deve estar presente neste dicionário,
o \emph{Opus} irá injetar o objeto da \emph{Interface} neste dicionário, permitindo que o \emph{Driver} a acesse.
