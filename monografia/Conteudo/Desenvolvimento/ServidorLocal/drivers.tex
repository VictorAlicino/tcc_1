\subsection{\textbf{Drivers}}

Sem uma padronização da comunicação entre dispositivos IoT, cada fabricante pode adotar protocolos distintos e estruturas variadas de 
mensagens. Diante deste cenário, foi adotado o conceito de \emph{Driver}s, criado para abstrair e padronizar a comunicação entre 
dispositivos e o \emph{Opus}.

\emph{Driver}s são pacotes escritos em Python e carregados no momento da inicialização do \emph{Opus}.
Sua função é atuar como um tradutor para um tipo específico de dispositivo, garantindo que seus comandos e dados sejam corretamente 
interpretados e processados. Além disso, caso o método utilizado pelo \emph{Driver} suporte descoberta automática, ele pode relatar novos 
dispositivos encontrados. Por conta disso, é necessário que \emph{Driver}s e o módulo de gerenciamento de dispositivos do \emph{Opus} trabalhem 
em conjunto, garantindo a integração de uma ampla gama de dispositivos sem a necessidade de modificar os módulos do sistema.
Como resultado, o sistema se torna modular e escalável.

Quando um dispositivo utiliza um protocolo não suportado nativamente pelo Python, como aqueles que requerem uma camada física diferente do TCP/IP 
(exemplo: comunicação Serial), ou que utilizam TCP/IP sem suporte HTTP (exemplo: MQTT), o \emph{Driver} pode acessar as 
\emph{Interfaces} para realizar a comunicação.

\begin{figure}[H]
    \conteudoFigura
    [Elaborado pelo Autor (2025)]
    {0.19}
    {driver_example.png}
    {Exemplo de Driver}
    {fig:driver-example}
\end{figure}

Para ser considerado um \emph{Driver}, o pacote deve seguir um padrão de nomenclatura e implementar algumas funções esperadas.
O pacote e o módulo principal devem ter o mesmo nome, pois o Opus busca pelo nome do \emph{Driver} dentro do diretório de \emph{Drivers}.
Caso um \emph{Driver} chamado ``exemplo'' seja encontrado no diretório de \emph{Drivers}, o Opus carregará o módulo principal do pacote, localizado em 
``drivers/exemplo/exemplo.py``. O módulo principal deve conter uma função ``\lstinline{start()}'' sem retornos, que será chamada no momento em que o 
\emph{Driver} for carregado.Além disso, deve possuir um dicionário global de \emph{Interfaces} com a seguinte assinatura: 
``\lstinline{interfaces: dict[str, Any]}''. Caso o \emph{Driver} necessite de uma \emph{Interface}, seu nome deve estar presente nesse dicionário,
e o \emph{Opus} injetará o objeto correspondente, permitindo que o \emph{Driver} o acesse.

Como \emph{Driver}s comandam toda a comunicação com os dispositivos, toda a estrutura de um dispositivo, assim como a lógica de controle, 
ficam a cargo do \emph{Driver}. Eles são autônomos e podem funcionar como aplicações independentes, permitindo a implementação de lógicas específicas,
independentemente de sua complexidade. Todavia, é importante que o \emph{Driver} siga o padrão esperado pelo \emph{Opus} em classes
voltadas para a comunicação com dispositivos.

\subsubsection{A Classe de Dispositivos}
A forma mais simples de abstrair um dispositivo é via uma classe.\ \emph{Driver}s têm liberdade total para implementar a classe dos dispositivos,
sendo necessário apenas que essa classe herde da classe genérica de dispositivos do \emph{Opus}, que contém os métodos esperados para 
o controle dos dispositivos através da aplicação. 

A classe genérica ``\lstinline{OpusDevice}'' é utilizada para representar dispositivos recém-descobertos, cujo tipo ainda não foi determinado. 
Após a identificação, o dispositivo deve ser instanciado como um tipo mais específico, herdado de ``\lstinline{OpusDevice}''. 

\subsubsection{O Registrando de Dispositivos}\label{sec:registrando-dispositivos-driver}
Durante a etapa de registro de dispositivos descobertos, o \emph{Opus} irá buscar pela função que retorna uma instância daquele tipo de dispositivo para o
\emph{Driver} em questão. O nome desta função deve ser ``\lstinline{new_}'' seguido do tipo do dispositivo. 

Por exemplo, para um dispositivo do tipo ``\lstinline{light}'' no \emph{Driver} de exemplo, chamado Driver1, a assinatura da função seria:

\begin{lstlisting}[language=Python]
def new_light() -> Driver1Light:
    return Driver1Light()
\end{lstlisting}

Um exemplo de um \emph{Driver} que implementa um dispositivo teórico do tipo ``\lstinline{light}'', com os requisitos mínimos é mostrado 
na figura~\ref{fig:driver-example}.