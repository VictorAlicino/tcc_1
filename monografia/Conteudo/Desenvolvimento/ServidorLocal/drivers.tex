\subsection{Drivers}

Sem uma padronização da comunicação entre dispositivos IoT, cada fabricante pode adotar protocolos distintos e estruturas variadas de 
mensagens. Diante deste cenário, foi adotado o conceito de \emph{Driver}s, criado para abstrair e padronizar a comunicação entre 
dispositivos e o \emph{Opus}.

\emph{Driver}s são pacotes escritos em Python e carregados no momento da inicialização do \emph{Opus},
sua função é atuar como um tradutor para um tipo específico de dispositivo, garantindo que 
comandos e dados sejam corretamente interpretados e processados. Além de, se suportado pelo método usado pelo driver, reportar
novos dispositivos encontrados. Por conta disso, é necessário que \emph{Drivers} e o módulo de gerenciamento de dispositivos do 
\emph{Opus} trabalhem em conjunto, garantindo a integração de uma ampla gama de dispositivos 
sem a necessidade de modificar os módulos do sistema, como resultado o sistema se torna modular e escalável.

Em casos onde o dispositivo se comunica via um protocolo não suportado nativamente pelo Python, como dispositivos que 
requerem uma camada física diferente das utilizadas pelo protocolo TCP/IP, como Serial, ou até mesmo que utilizam TCP/IP, 
porém não HTTP, como o MQTT, é possível que \emph{Driver}s acessem as \emph{Interfaces} para realizar a comunicação com o dispositivo.

\begin{figure}[h!]
    \conteudoFigura
    [Elaborado pelo Autor (2025)]
    {0.2}
    {driver_example.png}
    {Exemplo de Driver}
    {fig:driver-example}
\end{figure}

Para ser considerado um \emph{Driver}, o pacote deve seguir um padrão de nomenclatura e implementar algumas funções esperadas.
O pacote e o módulo principal devem ter o mesmo nome; o módulo principal deve ter implementado uma função ``\lstinline{start()}'' sem retornos,
essa função será chamada no momento em que o \emph{Driver} for carregado; o módulo principal deve ter um dicionário global de 
\emph{Interfaces} com a seguinte assinatura: \lstinline{interfaces: dict[str, Any]}, em caso do \emph{Driver} necessitar de uma \emph{Interface},
o nome dela deve estar presente neste dicionário e o \emph{Opus} injetará o objeto da \emph{Interface} neste dicionário,
permitindo que o \emph{Driver} a acesse.

Como \emph{Drivers} comandam toda a comunicação com os dispositivos, toda a estrutura de um dispositivo, assim como a lógica de controle
ficam a cargo do \emph{Driver}. Eles são autônomos e podem funcionar como aplicações independentes, dando liberdade de implementar a lógicas
singulares não importando sua complexidade, todavia, é importante que o \emph{Driver} siga o padrão esperado pelo \emph{Opus} em classes
voltadas para a comunicação com dispositivos.

\subsubsection{A Classe de Dispositivos}
A forma mais simples de abstrair um dispositivo é via uma classe.\ \emph{Drivers} tem liberdade total para implementar a classe dos dispositivos,
sendo necessário apenas que essa classe seja herdeira da classe genérica de dispositivos do \emph{Opus}, que contém os métodos esperados para 
o controle dos dispositivos através da aplicação.




%o módulo deve ter implementado uma função ``\lstinline{start()}'' sem retornos, essa função será 
%chamada no momento em que o \emph{Driver} for carregado, também deve ter um dicionário global de \emph{Interface}s com a seguinte assinatura
%\lstinline{interfaces: dict[str, Any]}, em caso do \emph{Driver} necessitar de uma \emph{Interface}, o nome dela deve estar presente neste dicionário,
%o \emph{Opus} irá injetar o objeto da \emph{Interface} neste dicionário, permitindo que o \emph{Driver} a acesse.
%
%A função ``\lstinline{start()}'' recebe os módulos do \emph{Opus} como argumentos, liberando acesso a todas as funcionalidades do sistema
%para o \emph{Driver}, isto é necessário para acessar os métodos de criação e manipulação de dispositivos, presentes no 
%módulo ``\lstinline{managers.devices}''.
