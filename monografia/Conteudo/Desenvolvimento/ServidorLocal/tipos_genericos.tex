\subsection{Registro e Aquisição de Dispositivos}

\begin{figure}[H]
    \conteudoFigura
    [Elaborado pelo Autor (2025)]
    {0.2}
    {opus_device_generic.png}
    {Classe Genérica de Dispositivo do Opus}
    {fig:generic-device-opus}
\end{figure}

Para garantir modularidade e flexibilidade na integração de novos dispositivos, o \emph{Opus} adota um modelo baseado em classes genéricas,
que servem como base para representar diferentes tipos de dispositivos e padronizar a interação com os \emph{Drivers}. Ao criar suas próprias
classes genéricas de dispositivo. Na figura~\ref{fig:generic-device-opus} é possível visualizar a classe
genérica de dispositivo do \emph{Opus}, já na figura \ref{fig:generic-device-tasmota} é possível ver uma implementação de uma classe
genérica de dispositivo implementada pelo \emph{Driver} Tasmota.

Observe que ambas as classes não possuem métodos de controle específicos, apenas métodos auxiliares, já que não é esperado que a instância dessa
classe seja utilizada para controlar os dispositivos, apenas representá-los.

\begin{figure}[H]
    \conteudoFigura
    [Elaborado pelo Autor (2025)]
    {0.2}
    {tasmota_device_generic.png}
    {Classe Genérica de Dispositivo do Driver Tasmota}
    {fig:generic-device-tasmota}
\end{figure}

Uma vez que dispositivo é registrado, o \emph{Opus} busca pelo método ``\lstinline{new_?()}'' [\ref{sec:registrando-dispositivos-driver}]
no \emph{Driver} correspondente, onde então, o \emph{Driver} deve retornar uma instância do dispositivo com seu tipo específico.

A figura~\ref{fig:class-light-generic} mostra um exemplo de uma classe genérica de dispositivo do tipo luz, que herda da classe genérica de dispositivo
do \emph{Opus}. Já a figura~\ref{fig:class-light-sonoff} mostra uma implementação de uma classe de dispositivo do tipo luz,
específica para o \emph{Driver} Sonoff, que implementa os métodos abstratos para a comunicação com o dispositivo.

\begin{figure}[H]
    \conteudoFigura
    [Elaborado pelo Autor (2025)]
    {0.15}
    {opus_light_generic_device.png}
    {Classe Genérica de Dispositivo do Tipo Luz}
    {fig:class-light-generic}
\end{figure}

\begin{figure}[H]
    \conteudoFigura
    [Elaborado pelo Autor (2025)]
    {0.15}
    {sonoff_light.png}
    {Classe de Dispositivo do Tipo Luz do Driver Sonoff}
    {fig:class-light-sonoff}
\end{figure}


\paragraph{Registro de Dispositivos}
A fim de oferecer suporte a dispositivos IoT que podem ser descobertos automaticamente por meio de sistemas de anúncio,
o \emph{Opus} organiza os dispositivos em dois estados distintos: \emph{Dispositivos Disponíveis} e \emph{Dispositivos Registrados}.

\emph{Dispositivos Disponíveis} são dispositivos detectados pelos \emph{Drivers}, mas que ainda não possuem uma definição
específica dentro do sistema. Eles herdam da classe base \texttt{OpusDevice} e representam unicamente a presença de um novo
dispositivo compatível, sem que haja, nesse momento, qualquer categorização funcional (como luz, ar-condicionado ou fechadura),
configuração ou armazenamento no banco de dados.

Para que um \emph{Dispositivo Disponível} possa ser utilizado ativamente no sistema, ele deve passar pelo processo de \emph{Registro}.
Esse processo consiste em atribuir ao dispositivo um tipo específico, vinculá-lo ao banco de dados do \emph{Opus} e habilitar seu controle
dentro da plataforma. Uma vez registrado, o dispositivo torna-se um \emph{Dispositivo Registrado}, adquirindo uma identidade
única dentro do \emph{Opus} e tornando-se acessível para controle e gerenciamento pelos usuários, e também gravando suas informações
no banco de dados local.