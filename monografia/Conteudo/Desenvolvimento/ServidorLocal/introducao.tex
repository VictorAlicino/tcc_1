\section{\textbf{Servidor Local}}
~\label{opus}

O controle de dispositivos, gerenciamento de permissões de usuário e abstração da estrutura de um edifício,
são responsabilidades assumidas pelo Servidor Local, chamado de \emph{Opus}.

O \emph{Opus} é uma aplicação escrita em Python visando convergir a operação de dispositivos IoT de vários
fabricantes e protocolos em um único ponto de controle, tal funcionalidade é obtida a partir da natureza modular
do Python que foi aqui aproveitada para criar partes expansíveis do sistema, chamadas de \emph{drivers} e \emph{interfaces}.

A aplicação é organizada em quatro módulos principais, como ilustrado na figura \ref{fig:opus1}, são eles:
\begin{figure}[h!]
    \conteudoFigura
    [Elaborado pelo Autor (2025)]
    {0.35}
    {opus_overall.png}
    {Módulos do Opus}
    {fig:opus1}
\end{figure}

\emph{Interfaces}, módulos que adicionam uma forma de comunicação ao \emph{Opus}, como, por exemplo, uma \emph{interface} para comunicação com dispositivos ZigBee\ref{sec:zigbee}, ou KNX\ref{sec:knx};
\emph{Drivers}, responsáveis por traduzir os comandos do \emph{Opus} para o protocolo do dispositivo, podendo ou não utilizar alguma \emph{interface} própria para esta finalidade;
\emph{Managers}, são módulos da aplicação que gerenciam seu funcionamento, recebem, enviam e processam comandos vindos do Servidor Remoto e por fim,
o banco de dados.

Módulos como \emph{drivers} e \emph{interfaces} terão seus objetos carregados no ambiente de execução do \emph{Opus}
via um dicionário, onde a chave é o nome do módulo e o valor é objeto.
Sendo assim para acessar um \emph{driver} ou \emph{interface} basta acessar o dicionário com o nome do módulo como no exemplo
a seguir:
\begin{lstlisting}[language=Python]
    interfaces['nome_da_interface'].metodo_desejado()
\end{lstlisting}

\subsection{Interfaces}
A fim de ser uma aplicação modular, que pode se comunicar com diversos dispositivos, foi necessário levar em conta a vasta 
gama de protocolos existentes hoje no mercado, sejam de dispositivos IoT com fins de automação residencial, ou equipamentos industriais
de automação de edifícios. Para preencher essa lacuna, foi criado o conceito de \emph{Interfaces}.

