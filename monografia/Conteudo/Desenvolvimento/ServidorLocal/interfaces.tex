\subsection{Interfaces}
\label{opus-sec:interfaces}

A fim de ser uma aplicação modular, que pode se comunicar com diversos dispositivos, foi necessário levar em conta a vasta 
gama de protocolos existentes hoje no mercado, sejam de dispositivos IoT com fins de automação residencial, ou equipamentos industriais
de automação de edifícios. Para preencher essa lacuna, foi criado o conceito de \emph{Interfaces}. Este conceito permite que novos meios de
comunicação sejam adicionados sem a necessidade de alterar o código-fonte dos módulos já existentes, criando um sistema expansível.
\begin{figure}[h!]
    \conteudoFigura
    [Elaborado pelo Autor (2025)]
    {0.2}
    {interface_example.png}
    {Exemplo de Interface}
    {fig:interface-example}
\end{figure}

Uma \emph{Interface} é um módulo Python que implementa um conjunto de métodos necessários para a comunicação com um protocolo específico. Cada interface
encapsula a lógica de comunicação para um método de comunicação e expõe aos módulos somente os métodos necessários para realizar a comunicação.
Todas as \emph{Interface}s recebem um dicionário de configurações, onde o usuário pode especificar parâmetros para a comunicação, desta forma
cada método de comunicação pode ser configurado conforme as necessidades do usuário e de forma dinâmica.

Para um módulo Python ser considerado uma \emph{Interface} pelo \emph{Opus}, ele deve implementar uma função ``\lstinline{initialize()}'' que retorna o objeto 
principal da classe implementada no módulo, este objeto será adicionado ao ambiente de execução do \emph{Opus} e estará então acessível para todos os módulos 
da aplicação. Também é necessário a implementação de um método ``\lstinline{begin(self, config: dict) -> bool}'' membro da classe principal, este método 
receberá o dicionário de configurações da \emph{Interface} e então começar a comunicação com o protocolo especificado. 
A figura~\ref{fig:interface-example} apresenta um exemplo prático da implementação de uma \emph{Interface}, destacando os métodos e função 
essenciais para o funcionamento.

As \emph{Interface}s são carregadas na aplicação em tempo de execução, de forma dinâmica no momento da inicialização dos módulos. Para que uma \emph{Interface}
seja carregada, é necessário que seu nome esteja listado no arquivo de configuração do \emph{Opus} seguido de um identificador 
único (exemplo: ``\lstinline{interface1<identificador>}''), como também é necessário que o arquivo contendo a implementação da \emph{Interface} esteja 
no diretório de \emph{Interface}s com o mesmo nome mencionado no arquivo de configuração (exemplo: ``\lstinline{./interfaces/interface1.py}''), em 
caso da mesma \emph{Interface} ser mencionada mais de uma vez no arquivo de configuração com o identificador diferente, será carregada uma instância
diferente da \emph{Interface} para cada identificador. Por fim, se ocorrer uma falha durante a inicialização de uma \emph{Interface} o \emph{Opus} 
não prosseguirá e encerrará a execução com um erro marcado como crítico no arquivo de log, sendo necessário corrigir o erro e reiniciar a aplicação.
