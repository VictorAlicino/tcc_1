\section{\textbf{Servidor Local}}
~\label{opus}

O controle de dispositivos, gerenciamento de permissões de usuário e abstração da estrutura de um edifício,
são responsabilidades assumidas pelo Servidor Local, chamado de \emph{Opus}.

O \emph{Opus} é uma aplicação escrita em Python visando convergir a operação de dispositivos IoT de vários
fabricantes e protocolos em um único ponto de controle, tal funcionalidade é obtida a partir da natureza modular
do Python que foi aqui aproveitada para criar partes expansíveis do sistema, chamadas de \emph{drivers} e \emph{interfaces}.

A aplicação é organizada em quatro módulos principais, como ilustrado na figura~\ref{fig:opus1}, são eles:
\begin{figure}[h!]
    \conteudoFigura
    [Elaborado pelo Autor (2025)]
    {0.35}
    {opus_overall.png}
    {Módulos do Opus}
    {fig:opus1}
\end{figure}

\emph{Interfaces}, módulos que adicionam uma forma de comunicação ao \emph{Opus}, como, por exemplo, uma \emph{interface} para comunicação com 
dispositivos ZigBee[\ref{sec:zigbee}], ou KNX[\ref{sec:knx}]; \emph{Drivers}, responsáveis por traduzir os comandos do \emph{Opus} para o 
protocolo do dispositivo, podendo ou não utilizar alguma \emph{interface} própria para esta finalidade; \emph{Managers}, são módulos da aplicação 
que gerenciam seu funcionamento, recebem, enviam e processam comandos vindos do Servidor Remoto e por fim, o banco de dados.

\section{Interfaces}
\label{opus-sec:interfaces}

O módulo de interfaces contém as implementações para comunicações externas do \emph{Opus} utilizando outros protocolos,
alguns \emph{drivers} podem requesitar que uma ou mais interfaces específicas estejam ativadas para que ele funcione corretamente.
O programa principal irá carregar cada interface cujo o nome do script estiver listado no arquivo de configuração, esse processo
se dará através de uma função '\lstinline{def initialize() -> Any}' que deve ser implementada em todas as interfaces, essa função
deverá retornar o objeto principal da classe desta interface, para que ele seja adicionado no ambiente de execução do \emph{Opus}.

\subsection{Interface MQTT}
\label{opus-sec:interfaces-mqtt}
A interface MQTT é implementada por padrão no \emph{Opus} uma vez que a comunicação do \emph{Opus} com o servidor \emph{Maestro}
é feita via MQTT, logo, ela é essencial para o funcionamento do sistema.

A implementação da interface MQTT segue uma abordagem simples para um script Python, apenas uma classe feita no padrão de 
projeto Singleton, isso não se fez tão necessário já que interfaces são carregadas para o ambiente de execução do 
\emph{Opus} por apenas um objeto que é compartilhado entre toda a aplicação, mas foi adotado por boas práticas.

\subsection{Drivers}

Os \emph{drivers} são responsáveis por traduzir os comandos do sistema para os dispositivos, de forma que cada dispositivo no \emph{Opus} está
associado a um \emph{driver} específico, desta forma, quando o usuário pedir que um determinado dispositivo seja ligado, por exemplo,
esse comando chamará o método correspondente da classe do dispositivo que utilizará seu \emph{driver} para enviar o comando para o hardware.

Usando o Tasmota como exemplo, um dos comandos que é possível usar, é o de ligar um AVAC, para isso é necessário enviar uma mensagem
MQTT em uma estrutura de JSON para o tópico: ``\lstinline{cmnd/TOPICO MQTT DO DISPOSITIVO/IRHVAC}'', com o seguinte conteúdo:
\begin{lstlisting}[language=Python]
    {
    "Vendor": O fabricante do AVAC,
    "Power": AVAC ligado ou desligado,
    "Mode":  Modo do AVAC, como frio, calor, ventilação, etc,
    "FanSpeed": Velocidade do ventilador,
    "Temp": Temperatura desejada
    }
\end{lstlisting}

A função do driver é abstrair detalhes como estes e expor para aplicação apenas os comandos que ela espera ter, como ligar, desligar, etc.
Para ligar um AVAC Tasmota no \emph{Opus} utilizando este driver, a aplicação apenas chama o método ``\lstinline{on()}'' do objeto. O
método ``\lstinline{on()}'' está ilustrado na figura \ref{fig:driver1}.
\begin{figure}[h!]
    \conteudoFigura
    [Elaborado pelo Autor (2024)]
    {0.35}
    {driver1.png}
    {Método \lstinline{on()} de dispositivos AVAC com o \emph{driver} Tasmota}
    {fig:driver1}
\end{figure}

