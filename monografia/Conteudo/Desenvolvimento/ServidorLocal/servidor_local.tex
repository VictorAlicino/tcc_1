\section{\textbf{Servidor Local}}
~\label{opus}

O controle de dispositivos, gerenciamento de permissões de usuário e abstração da estrutura de um edifício,
são responsabilidades assumidas pelo Servidor Local, chamado de \emph{Opus}.

O \emph{Opus} é uma aplicação escrita em Python visando convergir a operação de dispositivos IoT de vários
fabricantes e protocolos em um único ponto de controle, tal funcionalidade é obtida a partir da natureza modular
do Python que foi aqui aproveitada para criar partes expansíveis do sistema, chamadas de \emph{drivers} e \emph{interfaces}.

A aplicação é organizada em quatro módulos principais, como ilustrado na figura~\ref{fig:opus1}, são eles:
\begin{figure}[h!]
    \conteudoFigura
    [Elaborado pelo Autor (2025)]
    {0.35}
    {opus_overall.png}
    {Módulos do Opus}
    {fig:opus1}
\end{figure}

\textbf{Interfaces}, módulos que adicionam uma forma de comunicação ao \emph{Opus}, como, por exemplo, 
uma \emph{interface} para comunicação com dispositivos ZigBee[\ref{sec:zigbee}], ou KNX[\ref{sec:knx}]; 
\textbf{Drivers}, responsáveis por traduzir os comandos do \emph{Opus} para o protocolo do dispositivo, 
podendo ou não utilizar alguma \emph{interface} própria para esta finalidade; \textbf{Managers}, são módulos da aplicação 
que gerenciam seu funcionamento, recebem, enviam e processam comandos vindos do Servidor Remoto e por fim, o banco de dados.

\subsection{Arquivo de Configuração}
\label{sec:arquivo-configuracao}

Afim de personalizar a execução do \emph{Opus}, um arquivo de configuração é utilizado para definir parâmetros de execução, como
interfaces, drivers e configurações específicas de cada um. O arquivo de configuração é escrito em YAML, um formato de fácil leitura
e escrita, que permite a definição de estruturas de dados complexas de forma simples e legível.

O arquivo de configuração é dividido em seções, cada uma com um propósito específico, como interfaces, drivers e configurações gerais.
Na figura~\ref{fig:config-example} é possível ver um exemplo de arquivo de configuração do \emph{Opus}.

\begin{figure}[H]
    \conteudoFigura
    [Elaborado pelo Autor (2025)]
    {0.3}
    {config_file_example.png}
    {Exemplo de Arquivo de Configuração}
    {fig:config-example}
\end{figure}

\section{Interfaces}
\label{opus-sec:interfaces}

O módulo de interfaces contém as implementações para comunicações externas do \emph{Opus} utilizando outros protocolos,
alguns \emph{drivers} podem requesitar que uma ou mais interfaces específicas estejam ativadas para que ele funcione corretamente.
O programa principal irá carregar cada interface cujo o nome do script estiver listado no arquivo de configuração, esse processo
se dará através de uma função '\lstinline{def initialize() -> Any}' que deve ser implementada em todas as interfaces, essa função
deverá retornar o objeto principal da classe desta interface, para que ele seja adicionado no ambiente de execução do \emph{Opus}.

\subsection{Interface MQTT}
\label{opus-sec:interfaces-mqtt}
A interface MQTT é implementada por padrão no \emph{Opus} uma vez que a comunicação do \emph{Opus} com o servidor \emph{Maestro}
é feita via MQTT, logo, ela é essencial para o funcionamento do sistema.

A implementação da interface MQTT segue uma abordagem simples para um script Python, apenas uma classe feita no padrão de 
projeto Singleton, isso não se fez tão necessário já que interfaces são carregadas para o ambiente de execução do 
\emph{Opus} por apenas um objeto que é compartilhado entre toda a aplicação, mas foi adotado por boas práticas.

\subsection{Drivers}

Os \emph{drivers} são responsáveis por traduzir os comandos do sistema para os dispositivos, de forma que cada dispositivo no \emph{Opus} está
associado a um \emph{driver} específico, desta forma, quando o usuário pedir que um determinado dispositivo seja ligado, por exemplo,
esse comando chamará o método correspondente da classe do dispositivo que utilizará seu \emph{driver} para enviar o comando para o hardware.

Usando o Tasmota como exemplo, um dos comandos que é possível usar, é o de ligar um AVAC, para isso é necessário enviar uma mensagem
MQTT em uma estrutura de JSON para o tópico: ``\lstinline{cmnd/TOPICO MQTT DO DISPOSITIVO/IRHVAC}'', com o seguinte conteúdo:
\begin{lstlisting}[language=Python]
    {
    "Vendor": O fabricante do AVAC,
    "Power": AVAC ligado ou desligado,
    "Mode":  Modo do AVAC, como frio, calor, ventilação, etc,
    "FanSpeed": Velocidade do ventilador,
    "Temp": Temperatura desejada
    }
\end{lstlisting}

A função do driver é abstrair detalhes como estes e expor para aplicação apenas os comandos que ela espera ter, como ligar, desligar, etc.
Para ligar um AVAC Tasmota no \emph{Opus} utilizando este driver, a aplicação apenas chama o método ``\lstinline{on()}'' do objeto. O
método ``\lstinline{on()}'' está ilustrado na figura \ref{fig:driver1}.
\begin{figure}[h!]
    \conteudoFigura
    [Elaborado pelo Autor (2024)]
    {0.35}
    {driver1.png}
    {Método \lstinline{on()} de dispositivos AVAC com o \emph{driver} Tasmota}
    {fig:driver1}
\end{figure}



\subsection{\textbf{Gerenciadores}}

No ``núcleo'' do \emph{Opus}, estão os \emph{Managers}, ou Gerenciadores, eles são responsáveis por coordenar o fluxo
da aplicação \emph{Opus}, se comunicando utilizando a \emph{Interface} MQTT. Divididos em quatro módulos principais:
\begin{itemize}
    \item \lstinline{CloudManager}, que gerencia a comunicação com o \emph{Maestro};
    \item \lstinline{UserManager}, que gerencia os usuários do \emph{Opus} em questão.
    \item \lstinline{LocationManager}, que gerencia a estrutura do edifício no sistema;
    \item \lstinline{DeviceManager}, que gerencia os dispositivos;
\end{itemize}

\subsubsection{CloudManager}
\emph{CloudManager} é o módulo responsável pela comunicação entre o Servidor Local (\emph{Opus}) e Servidor Remoto (\emph{Maestro}), 
garantindo a sincronização de informações e o controle remoto dos dispositivos cadastrados no sistema. 
Essa comunicação é realizada principalmente via MQTT, permitindo que o \emph{Opus} envie e receba mensagens do Maestro de 
maneira assíncrona e eficiente. 

\paragraph{Funcionalidades Principais}
\begin{itemize}
    \item Login no Maestro: ao ser inicializado, o \emph{CloudManager} realiza um procedimento de login no Maestro, enviando informações como o 
        identificador do Servidor Local, dado pelo nome do Servidor Local definido pelo usuário no arquivo de configuração; seu endereço IP público
        e o tópico MQTT de \emph{callback}. Caso a autenticação falhe ou o Maestro não responda em um tempo limite, 
        a aplicação se encerra para evitar estados inconsistentes.
    \item Acesso das informações internas ao \emph{Opus}: o controle do que usuários podem ou não controlar é todo gerenciado por cada Servidor Local,
        em casos do \emph{Maestro} requerer informações sobre algum usuário, como quais dispositivos ele está autorizado a controlar, o \emph{CloudManager}
        irá fazer a integração com outros módulos do ``núcleo'' do \emph{Opus} para obter essas informações e enviar ao \emph{Maestro} via MQTT.
\end{itemize}


\subsubsection{UserManager}
\emph{UserManager} é o módulo responsável pelo gerenciamento de usuários dentro do \emph{Opus}, garantindo que apenas os usuários registrados e com as devidas
permissões possam acessar dispositivos cadastrados no sistema.

\paragraph{Funcionalidades Principais}
\begin{itemize}
    \item Sincronização de Usuários do Maestro: o \emph{UserManager} recebe informações de usuários cadastrados no Maestro e os atribui 
        localmente no \emph{Opus}. Esse processo inclui a atribuição daquele usuário a nível de acesso chamado de \emph{role} e a persistência dos dados
        no banco de dados local.
    \item Gerenciamento de Permissões: o módulo é responsável por checar níveis de permissão de usuários, entregando esse dados a outros módulos 
        como o \emph{DeviceManager} que utiliza essas informações para permitir ou não o acesso a um dispositivo.
    \item Comunicação via MQTT:\ o \emph{UserManager} também estabelece um canal de comunicação via MQTT com o Maestro, para troca de informações
        sobre usuários, como por exemplo, a atribuição de um novo usuário ao \emph{Opus}.
\end{itemize}



\subsubsection{LocationManager}
\emph{LocationManager} é o módulo responsável pela organização espacial dos dispositivos dentro do \emph{Opus},
permitindo estruturar o ambiente em uma hierarquia composta por prédios, espaços e salas. Essa estrutura foi pensada
para facilitar a implementação em vários tipos de edifícios, mantendo um balanço entre modularidade e simplicidade.

\paragraph{Funcionalidades Principais}
\begin{itemize}
    \item Gerenciamento da Estrutura do Edifício: o \emph{LocationManager} organiza os dispositivos do \emph{Opus} 
    em uma estrutura hierárquica composta por:
    \begin{itemize}
        \item Prédios (Buildings): representam grandes unidades físicas, como edifícios propriamente ditos, ou setores, como blocos de um campus;
        \item Espaços (Spaces): são subdivisões dentro dos edifícios, como andares em um prédio, sendo um agrupamento de salas;
        \item Salas (Rooms): representam ambientes específicos em um espaço, onde dispositivos podem ser instalados.
    \end{itemize}
    \item Criação e Registro de Estruturas: o módulo realiza toda a manutenção das estruturas no banco de dados local e também mantém as estruturas 
        em memória durante a execução.
    \item Carregamento de Estrutura Existentes: durante a inicialização, o \emph{LocationManager} consulta o banco de dados do
        \emph{Opus} e carrega todas as estruturas previamente registradas em memória, onde ocorrem todas as operações de gerenciamento.
    \item Configuração e Comunicação via MQTT: o módulo estabelece comunicação via MQTT, permitindo que o Maestro e
        outros clientes solicitem informações ou realizem operações.
        O \emph{LocationManager} processa mensagens em tópicos MQTT específicos para:
    \begin{itemize}
        \item Criar novos prédios, espaços e salas conforme solicitado.
        \item Fornecer a lista atual de locais cadastrados no sistema.
    \end{itemize}
\end{itemize}

\subsubsection{DeviceManager}
\emph{DeviceManager} é o módulo responsável pelo gerenciamento de dispositivos dentro do \emph{Opus}.
Ele controla a descoberta, registro, armazenamento e operação de dispositivos conectados, garantindo a integração entre diferentes drivers
e a comunicação entre o \emph{Opus} e o \emph{Maestro}.

\paragraph{Funcionalidades Principais}
\begin{itemize}
    \item Gerenciamento de Dispositivos: o \emph{DeviceManager} mantém um catálogo de dispositivos disponíveis e registrados em memória,
        esse catálogo é divido nos tipos de dispositivos suportados pelo \emph{Opus};
    \item Registro de Dispositivos: o módulo permite registrar dispositivos no sistema,
        vinculando cada um a um driver específico e a uma sala dentro da abstração dos edifícios cadastrados no \emph{Opus}. Esse processo inclui:
    \begin{itemize}
        \item Verificação de Disponibilidade: o dispositivo precisa estar previamente detectado na lista de dispositivos disponíveis, encontrados
            pelos \emph{Drivers};
        \item Verificação do Local: o dispositivo apenas pode ser registrado em salas previamente cadastradas
            no \emph{LocationManager};
        \item Criação no Banco de Dados: após a validação, o dispositivo é adicionado a lista de dispositivos registrados e também é
            adicionado ao banco de dados para persistência dos dados em caso de reinicialização do sistema.
    \end{itemize}
    \item Classificação por Tipo e Driver: o \emph{DeviceManager} categoriza os dispositivos de acordo com seu tipo e \emph{Driver}, para
        os fins deste trabalho, estão suportados os seguintes tipos de dispositivos:
    \begin{itemize}
        \item LIGHT: Dispositivos de iluminação controláveis;
        \item HVAC: Sistemas de Ar-Condicionado.
    \end{itemize}
        Outros dispositivos podem ser adicionados através da implementação de novos drivers.
    \item Comunicação via MQTT: o módulo assim como os outros gerenciadores, também está integrado ao sistema de mensagens MQTT,
        permitindo o controle remoto dos dispositivos via mensagens enviadas pelo \emph{Maestro}.
\end{itemize}

\subsection{\textbf{Banco de Dados}}
A escolha do banco de dados para o Servidor Local \emph{Opus} foi tomada a partir dos seguintes critérios:
\begin{itemize}
    \item Banco autocontido: Não requer a instalação de um servidor de banco de dados separado, 
        tornando a aplicação mais simples e portátil;
    \item Desempenho eficiente: Para um ambiente local, SQLite oferece operações rápidas de leitura e escrita, 
        atendendo bem às necessidades do sistema;
    \item Segurança dos dados: Com um banco de dados local, as informações de dispositivos de um edifícios permanacem restritas para 
        acesso externo.
\end{itemize}

Com base nesses critérios, foi escolhido o SQLite como banco de dados para o \emph{Opus}.
O SQLite é um banco de dados relacional que não requer um servidor separado para funcionar, também tem uma boa 
integração com Python através do ORM SQLAlchemy, o que facilita a modelagem e manipulação dos dados.

\subsubsection{Estrutura do Banco de Dados}
O banco de dados do \emph{Opus} segue um modelo relacional estruturado para representar a hierarquia de um edifício inteligente,
usuários, dispositivos registrados e permissões de acesso. A estrutura inclui as seguintes tabelas principais:
\begin{itemize}
    \item Building: Representa os edifícios cadastrados no sistema;
    \item BuildingSpace: Define espaços dentro dos edifícios, como andares em um prédio;
    \item BuildingRoom: Representa salas dentro dos espaços;
    \item Device: Contém os dispositivos IoT cadastrados, armazenando informações como tipo de dispositivo, 
        nome e dados específicos de configuração que variam de acordo com o \emph{Driver} utilizado;
    \item Role: Define os diferentes papéis dos usuários no sistema, com níveis de segurança específicos;
    \item User: Armazena os usuários registrados;
    \item \texttt{role\_device} (tabela associativa): Relaciona os dispositivos que cada papel pode acessar.
\end{itemize}

A modelagem utiliza UUIDs como chaves primárias para garantir identificadores únicos globalmente.

O ERD (Diagrama de Relacionamento de Entidades) do banco de dados do \emph{Opus} é mostrado na figura~\ref{fig:opus-erd}.
\begin{figure}[H]
    \conteudoFigura
    [Elaborado pelo Autor (2025)]
    {0.3}
    {opus_erd.PNG}
    {Diagrama de Relacionamento de Entidades do Banco de Dados do Opus}
    {fig:opus-erd}
\end{figure}
O diagrama da figura~\ref{fig:opus-erd} ilustra muito bem como a hierarquia de um edifício é representada no banco de dados,
dispositivos são contidos em salas que são contidas em espaços que são contidos em prédios.
Um ponto importante a ser observado é a relação entre a tabela \texttt{role} e a tabela \texttt{device},
que é feita através da tabela associativa \texttt{role\_device}. \texttt{role} define os níveis de acesso dos usuários,
cada usuário tem uma \texttt{role} que é representada por um valor numérico onde quanto menor o valor, maior o nível de acesso,
sendo 0 o nível de administrador. A tabela associativa \texttt{role\_device} relaciona os dispositivos que cada \texttt{role}
pode acessar, isso garante que o acesso a cada dispositivo seja controlado de acordo com o nível de permissão do usuário.

O banco de dados do \emph{Opus} tem as seguintes informações em cada tabela:
\begin{itemize}
    \item \textbf{Building}  
      \begin{itemize}
          \item \texttt{building\_pk: UUID} - Identificador único do edifício;
          \item \texttt{building\_name: str} - Nome do edifício.
      \end{itemize}
    \item \textbf{BuildingSpace}  
      \begin{itemize}
          \item \texttt{building\_space\_pk: UUID} - Identificador único do espaço;
          \item \texttt{building\_fk: UUID} - Chave estrangeira para o edifício;
          \item \texttt{space\_name: str} - Nome do espaço.
      \end{itemize}
    \item \textbf{BuildingRoom}  
      \begin{itemize}
          \item \texttt{building\_room\_pk: UUID} - Identificador único da sala;
          \item \texttt{building\_space\_fk: UUID} - Chave estrangeira para o espaço;
          \item \texttt{room\_name: str} - Nome da sala.
      \end{itemize}
    \item \textbf{Device}  
      \begin{itemize}
          \item \texttt{device\_pk: UUID} - Identificador único do dispositivo;
          \item \texttt{room\_fk: UUID} - Chave estrangeira para a sala;
          \item \texttt{device\_name: str} - Nome do dispositivo;
          \item \texttt{device\_type: str} - Tipo do dispositivo;
          \item \texttt{driver\_name: str} - Nome do \emph{Driver} utilizado;
          \item \texttt{driver\_data: JSON} - Dados específicos do \emph{Driver}.
      \end{itemize}
    \item \textbf{Role}  
      \begin{itemize}
          \item \texttt{role\_pk: UUID} - Identificador único do papel;
          \item \texttt{role\_name: str} - Nome do papel;
          \item \texttt{security\_level: int} - Nível de segurança do papel.
      \end{itemize}
    \item \textbf{User}  
      \begin{itemize}
          \item \texttt{user\_pk: UUID} - Identificador único do usuário;
          \item \texttt{given\_name: str} - Nome do usuário;
          \item \texttt{email: str} - E-mail do usuário;
          \item \texttt{fk\_role: UUID} - Chave estrangeira para o papel.
      \end{itemize}
\end{itemize}

\subsection{Registro e Aquisição de Dispositivos}

\begin{figure}[H]
    \conteudoFigura
    [Elaborado pelo Autor (2025)]
    {0.2}
    {opus_device_generic.png}
    {Classe Genérica de Dispositivo do Opus}
    {fig:generic-device-opus}
\end{figure}

Para garantir modularidade e flexibilidade na integração de novos dispositivos, o \emph{Opus} adota um modelo baseado em classes genéricas,
que servem como base para representar diferentes tipos de dispositivos e padronizar a interação com os \emph{Drivers}. Ao criar suas próprias
classes genéricas de dispositivo. Na figura~\ref{fig:generic-device-opus} é possível visualizar a classe
genérica de dispositivo do \emph{Opus}, já na figura \ref{fig:generic-device-tasmota} é possível ver uma implementação de uma classe
genérica de dispositivo implementada pelo \emph{Driver} Tasmota.

Observe que ambas as classes não possuem métodos de controle específicos, apenas métodos auxiliares, já que não é esperado que a instância dessa
classe seja utilizada para controlar os dispositivos, apenas representá-los.

\begin{figure}[H]
    \conteudoFigura
    [Elaborado pelo Autor (2025)]
    {0.2}
    {tasmota_device_generic.png}
    {Classe Genérica de Dispositivo do Driver Tasmota}
    {fig:generic-device-tasmota}
\end{figure}

Uma vez que dispositivo é registrado, o \emph{Opus} busca pelo método ``\lstinline{new_?()}'' [\ref{sec:registrando-dispositivos-driver}]
no \emph{Driver} correspondente, onde então, o \emph{Driver} deve retornar uma instância do dispositivo com seu tipo específico.

A figura~\ref{fig:class-light-generic} mostra um exemplo de uma classe genérica de dispositivo do tipo luz, que herda da classe genérica de dispositivo
do \emph{Opus}. Já a figura~\ref{fig:class-light-sonoff} mostra uma implementação de uma classe de dispositivo do tipo luz,
específica para o \emph{Driver} Sonoff, que implementa os métodos abstratos para a comunicação com o dispositivo.

\begin{figure}[H]
    \conteudoFigura
    [Elaborado pelo Autor (2025)]
    {0.15}
    {opus_light_generic_device.png}
    {Classe Genérica de Dispositivo do Tipo Luz}
    {fig:class-light-generic}
\end{figure}

\begin{figure}[H]
    \conteudoFigura
    [Elaborado pelo Autor (2025)]
    {0.15}
    {sonoff_light.png}
    {Classe de Dispositivo do Tipo Luz do Driver Sonoff}
    {fig:class-light-sonoff}
\end{figure}


\paragraph{Registro de Dispositivos}
A fim de oferecer suporte a dispositivos IoT que podem ser descobertos automaticamente por meio de sistemas de anúncio,
o \emph{Opus} organiza os dispositivos em dois estados distintos: \emph{Dispositivos Disponíveis} e \emph{Dispositivos Registrados}.

\emph{Dispositivos Disponíveis} são dispositivos detectados pelos \emph{Drivers}, mas que ainda não possuem uma definição
específica dentro do sistema. Eles herdam da classe base \texttt{OpusDevice} e representam unicamente a presença de um novo
dispositivo compatível, sem que haja, nesse momento, qualquer categorização funcional (como luz, ar-condicionado ou fechadura),
configuração ou armazenamento no banco de dados.

Para que um \emph{Dispositivo Disponível} possa ser utilizado ativamente no sistema, ele deve passar pelo processo de \emph{Registro}.
Esse processo consiste em atribuir ao dispositivo um tipo específico, vinculá-lo ao banco de dados do \emph{Opus} e habilitar seu controle
dentro da plataforma. Uma vez registrado, o dispositivo torna-se um \emph{Dispositivo Registrado}, adquirindo uma identidade
única dentro do \emph{Opus} e tornando-se acessível para controle e gerenciamento pelos usuários, e também gravando suas informações
no banco de dados local.
