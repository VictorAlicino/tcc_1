\section{Arquitetura}
\subsection{Camada Física}

A arquitetura proposta é ilustrada na figura \ref{fig:arquitetura1}.
\begin{figure}[h!]
    \conteudoFigura
    [Elaborado pelo Autor (2024)]
    {0.20}
    {overall1.png}
    {Arquitetura completa}
    {fig:arquitetura1}
\end{figure}

\label{arq-subsec:camada-fisica}
Camada física é o nome dado por este trabalho a tudo que tem por característica
principal, seu hardware, como é o caso de um microcontrolador e um smartphone.

Nesta camada estão dispostos os dispositivos controlados pelo \emph{Opus} e as interfaces
gráficas de usuário (GUI) como o aplicativo móvel e o aplicativo web.

\subsection{Dispositivos Controlados}
\label{arq-subsubsec:dispositivos-controlados}
O sistema \emph{Opus} é feito para ser possível controlar qualquer dispositivo que o desenvolvedor
projetar um \emph{driver} para ele. Esses dispositivos em sua forma física normalmente
são constituídos de um microcontrolador com acesso à internet e um ou mais atuadores.

\subsection{Interfaces Gráficas de Usuário}
\label{arq-subsubsec:interfaces-graficas-usuario}
As interfaces gráficas de usuário são os meios que o usuário tem para interagir com o sistema.
Nesta arquitetura temos duas interfaces gráficas de usuário: o aplicativo móvel e o aplicativo web.
O aplicativo móvel é feito para servir de ponto principal de acesso para o usuário, podendo interagir com
os \emph{Dispositivos Controlados} pelo \emph{Opus} já o aplicativo web é a forma que o administrador
tem de interagir com o \emph{Maestro}.


\section{Rede Local}
\label{arq-subsec:rede-local}
A rede local refere-se a conexão local dentro de um ambiente, como a rede interna de um edifício. Ela pode ou não
estar conectada a internet já que a maioria dos protocolos de comunicação utilizado pelos
\emph{Dispositivos Controlados} não requerem conexão com a internet, apenas à uma rede local com DHCP.

\subsection{Banco de Dados}
\label{arq-subsubsec:rede-local-banco-dados}
Dentro da rede local deve estar situado o banco de dados que o \emph{Opus} utiliza para armazenar a estrutura
do edifício e os dados dos \emph{Dispositivos Controlados}. Devido ao fato do \emph{Opus} ser toda uma aplicação 
contida dentro de si mesma, sem dependências além das bibliotecas utilizadas na sua concepção e das 
escolhidas pelo usuário como os \emph{Drivers}, o banco de dados escolhido foi o SQLite3, assim não é necessário
que o usuário precise instalar outro software para que o \emph{Opus} funcione.

\subsection{Opus}
\label{arq-subsubsec:rede-local-opus}
O \emph{Opus} é o núcleo do sistema, a aplicação propriamente dita, responsável por se comunicar
com os \emph{Dispositivos Controlados} e manter as estruturas do edifício. O \emph{Opus} requer comunicação
constante com o \emph{Maestro} para receber comandos do usuário, logo, apesar de estar situado na Rede Local, que
por sua vez não necessita de conexão com a internet, o \emph{Opus} necessita de uma.

\subsection{Drivers}
\label{arq-subsubsec:rede-local-drivers}
Os \emph{Drivers} são as partes do \emph{Opus} que podem ser estendidas por outros programadores,
\emph{Drivers} são responsáveis por se comunicar com os \emph{Dispositivos Controlados}, traduzindo
os comandos do \emph{Opus} para o protocolo do dispositivo, eles são desenvolvidos como módulos do 
Python, e são carregados dinamicamente pelo \emph{Opus}.

\subsection{Tasmota}
\label{arq-subsec:rede-local-tasmota}
Para provar a viabilidade do sistema, foi escrito um \emph{Driver} para o \emph{Opus},
um driver para comunicação com o firmware Tasmota \ref{sec:tasmota}.


\section{Internet}
\label{arq-subsec:internet}
Na camada de internet está o \emph{Maestro}, o servidor central na nuvem, onde ficam armazenados
os usuários e os servidores \emph{Opus} cadastrados.

\subsection{Banco de Dados}
\label{arq-subsubsec:internet-banco-dados}
O banco de dados da camada de internet é o responsável por armazenar os usuários e os servidores
\emph{Opus} cadastrados, este banco de dados não necessariamente precisa estar na mesma máquina
que o \emph{Maestro}, sendo assim, existe a possbilidade de configurar um banco de dados externo.
A escolha do banco de dados foi o PostgreSQL, por ser um banco de dados de código aberto e versátil.

\subsection{Autenticação do Google}
\label{arq-subsubsec:internet-autenticacao-google}
Inspirado no modelo \emph{Zero Trust} ou \emph{Confiança Zero} \cite{Kang2023}, o sistema não tenta implementar autenticação de usuários
e sim delega essa responsabilidade para uma entidade externa que pode verificar a autenticidade do usuário, 
aqui foi escolhido o serviço de autenticação da empresa Google.

\subsection{Maestro}
\label{arq-subsubsec:internet-maestro}
O \emph{Maestro} é o servidor central do sistema, responsável por manter os usuários e os servidores registrados,
além de ser o responsável por intermediar a comunicação entre os servidores \emph{Opus} e os usuários.
