\chapter{Sistema Multiusuários}
\label{chap:multiusuarios}

Seguindo com o objetivo do trabalho, com o sistema montado foi pensado em uma forma de tornar ele acessível para vários usuários em um mesmo ambiente
ao mesmo tempo. Para chegar a este objetivo foi desenvolvido o seguinte método:

O administrador de um Servidor \emph{Opus} pode requisitar através do \emph{Maestro} a criação de um código de barras bidimensional (QR Code) que será associado 
a um dispositivo. Outros usuários do \emph{Maestro} podem então escanear esse QR Code através de uma Interface de Usuário para o acesso ao dispositivo como 
visitante, mesmo sem ser um usuário associado ao Servidor \emph{Opus} em questão.

Com essa abordagem, são necessários limites bem definidos para não haver conflito, os limites definidos foram:
\begin{itemize}
    \item Escanear o QR Code concede ao visitante acesso ao dispositivo por tempo limitado, após o tempo expirar o visitante é desconectado automaticamente
        e precisa escanear o QR Code novamente para continuar a interação.
    \item Somente um visitante pode estar conectado a um dispositivo por vez, se um novo visitante escanear o QR Code o visitante anterior é desconectado.
\end{itemize}

\section{Requisitando um QR Code}
\label{sec:requisitando_qr_code}

Foi adicionado ao \emph{Maestro} uma nova funcionalidade que permite ao administrador de um Servidor \emph{Opus} requisitar um QR Code para um dispositivo.
Exposto na rota \texttt{/qr\_code/\emph{\{server\_id\}}/\emph{\{device\_id\}}} o usuário pode requisitar um QR Code para um dispositivo específico de um Servidor \emph{Opus},
o \emph{Maestro} então irá checar se o usuário está na lista de administradores do Servidor \emph{Opus}, ao confirmar, ele irá gerar um código de barras 
bidimensional QR Code contendo um JWT (\emph{JSON Web Token}) com as informações do dispositivo e servidor, esse QR Code é então salvo na pasta ``assets/''
do \emph{Maestro} e seu caminho é retornado ao usuário por um redirecionamento para a rota \texttt{/assets/\emph{\{qr\_code\_path\}}}.

\begin{figure}[h!]
    \conteudoFigura
    [Elaborado pelo Autor (2025)]
    {0.2}
    {qr_code_gen.png}
    {Implementação da função que gera o QR Code}
    {fig:qr_code_func}
\end{figure}

O trecho de código na figura~\ref{fig:qr_code_func} mostra a implementação da função que gera o QR Code.

\section{Escaneando o QR Code}
\label{sec:escaneando_qr_code}

Para ler o QR Code é necessário que o usuário possua o aplicativo \emph{Conductor} instalado em seu telefone. No aplicativo, foi adicionada uma tela
de leitura de QR Code que utiliza a biblioteca \emph{react-native-camera} para a leitura do QR Code, como mostrado na figura~\ref{fig:qr_code_scan}.

\begin{figure}[h!]
    \conteudoFigura
    [Elaborado pelo Autor (2025)]
    {0.15}
    {qr_code_scan.jpg}
    {Tela de leitura de QR Code no aplicativo \emph{Conductor}}
    {fig:qr_code_scan}
\end{figure}

Ao escanear o QR Code, o aplicativo irá extrair o JWT contido no QR Code e enviar para o \emph{Maestro} por uma requisição POST na rota
\texttt{/guest\_acces/\emph{\{cypher\_text\}}}, como as chamadas de API são autenticadas, o Servidor pode extrair o usuário fazendo a requisição
do acesso. O \emph{Maestro} então irá verificar se o JWT é válido, ou seja, checar se o Servidor \emph{Opus} presente no JWT existe.

Se o JWT for válido, o \emph{Maestro} então irá verificar se o usuário que requisitou o acesso, é o último requisitante ou um novo visitante,
isso é feito checando um dicionário de visitantes, contendo o ID do Servidor \emph{Opus}, o ID do dispositivo e o ID do atual usuário visitante,
a representação em JSON do dicionário é mostrada na figura~\ref{fig:guest_dict}.

\begin{figure}[h!]
    \conteudoFigura
    [Elaborado pelo Autor (2025)]
    {0.2}
    {guest_users_dict.png}
    {Representação em JSON do dicionário de visitantes}
    {fig:guest_dict}
\end{figure}

Caso, o usuário seja um novo visitante, ou seja, não é o mesmo usuário que escaneou o QR Code anteriormente, o \emph{Maestro} atualizar o dicionário,
com o ID do novo visitante e seu tempo de expiração, que é de trinta minutos, garantindo assim que os limites definidos sejam respeitados.

Caso o usuário seja o mesmo que escaneou o QR Code anteriormente, o \emph{Maestro} irá atualizar o tempo de expiração do visitante, garantindo
que o visitante possa continuar a interação com o dispositivo.

Por fim, o \emph{Maestro} retornará ao aplicativo \emph{Conductor} um JSON contendo o tempo de expiração do acesso, os dados do dispositivo e seu 
estado atual, o que é suficiente para o aplicativo \emph{Conductor} se conectar ao dispositivo.

