\chapter{Testes e Resultados}
\label{testes_e_resultados}

% % Objetivos Específicos	
% \subsection{Objetivos Específicos}
% 
% Para alcançar o objetivo do trabalho, é proposto uma aplicação simplificada que permita entender a viabilidade do uso de um 
% sistema de edifício inteligente por diversos usuários, focando no controle das unidades de climatização.
% 
% Dentre os principais objetivos específicos destacam-se:
% 
% \begin{itemize}
%     \item Realizar uma pesquisa bibliográfica sobre edifícios inteligentes e suas demandas;
%     \item Definir requisitos para um sistema de gerenciamento de um edifício inteligente;
%     \item Propor e implementar um sistema de gerenciamento para edifícios inteligentes;
%     \item Tornar este sistema acessível a múltiplos usuários sem necessidade de cadastro prévio;
% \end{itemize}

Para atingir o objetivo geral deste trabalho, foi desenvolvido o sistema \textbf{Opus} com suas três partes principais:
\emph{Opus} (Servidor Local), \emph{Maestro} (Servidor Remoto) e \emph{Conductor} (Interface de Usuário), o desenvolvimento do sistema se
ateve ao foco nas unidades de climatização e não foram desenvolvidas funcionalidades além das mínimas necessárias para o controle destas unidades,
desta forma, interfaces gráficas para comandos de manipulações do banco de dados do Servidor Local não foram implementadas, 
como definir quais níveis de acesso podem controlar quais dispositivos, sendo necessária a manipulação direta do banco de dados nesses casos.

\section{Materiais}

Para a realização dos testes, foram utilizados os seguintes materiais para cada parte do sistema \textbf{Opus}:

\textbf{Servidor Local}: para executar o \emph{Opus} foi utilizado um computador com o sistema operacional Windows 10; para o broker MQTT necessário
para a comunicação MQTT local, foi utilizado um Raspberry Pi 3 na mesma rede, executando o broker MQTT Mosquitto; Para aceitar os comandos do 
\emph{Opus} e convertê-los em sinais infravermelhos legíveis para um ar-condicionado, foi utilizado um dispositivo emissor de infravermelho inteligente,
modificado para suportar o firmware Tasmota.

\textbf{Servidor Remoto}: para executar o Servidor Remoto, foi necessário um servidor na nuvem, para isso foi utilizado uma instância na OCI
(Oracle Cloud Infrastructure) com o sistema operacional Ubuntu 22.04.5 LTS, um broker MQTT Mosquitto também foi configurado na mesma instância para 
a comunicação entre o \emph{Maestro} e o \emph{Conductor}.

\textbf{Interface de Usuário}: para a Interface de Usuário, foram utilizados três
smartphones que possuem cada um uma conta Google diferente para se autenticar no aplicativo \emph{Conductor}, um para cada usuário.

\subsection{Testes}

Tanto o \emph{Opus} quanto o \emph{Maestro} produzem logs de suas atividades, que foram utilizados para verificar se os comandos enviados
pela Interface de Usuário foram recebidos e processados corretamente.

\subsubsection{Teste 1}
O primeiro teste consistiu em verificar se o \emph{Maestro} é capaz de suportar três usuários utilizando a autenticação do Google enviada pelo
\emph{Conductor}.

\begin{figure}[h!]
    \conteudoFigura
    [Elaborado pelo Autor (2025)]
    {0.6}
    {teste1.png}
    {LOGs do \emph{Maestro} após o login dos usuários}
    {fig:test1}
\end{figure}

Como pode ser visto na Figura \ref{fig:test1}, o \emph{Maestro} foi capaz de autenticar os três usuários.

\subsubsection{Teste 2}
O segundo teste consistiu em verificar se o \emph{Maestro} é capaz de gerar um QR Code para um dispositivo. Como não há uma interface para 
acessar essa rota no \emph{Maestro}, foi necessário acessar diretamente a rota por um software de requisições HTTP.

Utilizando o software Insomnia, foi feita uma requisição GET para a rota:
\begin{lstlisting}
    /opus_server/qr_code/a05d2db1-d40b-4bdb-ab47-f243468f27b7/
    e3e9906f-fcb0-11ef-8ded-001a7dda710a
\end{lstlisting}
onde ``a05d2db1-d40b-4bdb-ab47-f243468f27b7'' 
é o ID do Servidor \emph{Opus} utilizado no teste e ``e3e9906f-fcb0-11ef-8ded-001a7dda710a'' é o ID do dispositivo com Tasmota no Servidor Local.

\begin{figure}[h!]
    \conteudoFigura
    [Elaborado pelo Autor (2025)]
    {0.5}
    {teste2.png}
    {QR Code gerado pelo \emph{Maestro}}
    {fig:test2}
\end{figure}

Como mostrado na Figura \ref{fig:maestro-qr}, o \emph{Maestro} foi capaz de gerar o QR Code corretamente.

\subsubsection{Teste 3}

O último teste consistiu em verificar se o \emph{Maestro} é capaz de gerenciar o acesso de um dispositivo para usuários visitantes, seguindo os limites
estabelecidos no capítulo~\ref{chap:multiusuarios} 

O teste consistiu em primeiro o usuário ``Victor'' escanear o QR Code gerado no teste 2 e então tentar controlar o dispositivo, o que foi feito com sucesso
como mostra os logs do \emph{Maestro} na Figura \ref{fig:test3-1} e do \emph{Opus} na Figura \ref{fig:test3-2}.

\begin{figure}[h!]
    \conteudoFigura
    [Elaborado pelo Autor (2025)]
    {0.35}
    {teste3_2.png}
    {LOGs do \emph{Maestro} 1}
    {fig:test3-1}
\end{figure}

\begin{figure}[h!]
    \conteudoFigura
    [Elaborado pelo Autor (2025)]
    {0.4}
    {teste3_1.png}
    {LOGs do \emph{Opus} 1}
    {fig:test3-2}
\end{figure}

Em seguida, o usuário ``Sergio'' escaneou o QR Code e tentou controlar o dispositivo, o que também foi feito com sucesso como mostra os
logs do \emph{Maestro} na Figura \ref{fig:test3-3} e do \emph{Opus} na Figura \ref{fig:test3-4}.

\begin{figure}[h!]
    \conteudoFigura
    [Elaborado pelo Autor (2025)]
    {0.35}
    {teste3_3.png}
    {LOGs do \emph{Maestro} 2}
    {fig:test3-3}
\end{figure}

\begin{figure}[h!]
    \conteudoFigura
    [Elaborado pelo Autor (2025)]
    {0.4}
    {teste3_4.png}
    {LOGs do \emph{Opus} 2}
    {fig:test3-4}
\end{figure}

Logo em seguida, o usuário ``Meyre'' escaneou o QR Code e o usuário ``Sergio'' tentou controlar o dispositivo, o que não foi possível, como esperado.

E por fim o usuário ``Meyre'' tentou controlar o dispositivo, o que foi feito com sucesso como mostra os logs do \emph{Opus} na Figura \ref{fig:test3-5}.

\begin{figure}[h!]
    \conteudoFigura
    [Elaborado pelo Autor (2025)]
    {0.4}
    {teste3_6.png}
    {LOGs do \emph{Opus} 3}
    {fig:test3-5}
\end{figure}

\section{Resultados}

% \subsection{Objetivo Geral}
% Este trabalho visa desenvolver um sistema básico que demonstre a viabilidade da operação simultânea de múltiplos usuários
% em um edifício inteligente. A proposta é criar uma solução funcional que atenda a múltiplos usuários em um ambiente compartilhado. 
% 
% % Objetivos Específicos	
% \subsection{Objetivos Específicos}
% 
% Para alcançar o objetivo do trabalho, é proposto uma aplicação simplificada que permita entender a viabilidade do uso de um 
% sistema de edifício inteligente por diversos usuários, focando no controle das unidades de climatização.
% 
% Dentre os principais objetivos específicos destacam-se:
% 
% \begin{itemize}
%     \item Realizar uma pesquisa bibliográfica sobre edifícios inteligentes e suas demandas;
%     \item Definir requisitos para um sistema de gerenciamento de um edifício inteligente;
%     \item Propor e implementar um sistema de gerenciamento para edifícios inteligentes;
%     \item Tornar este sistema acessível a múltiplos usuários sem necessidade de cadastro prévio;
% \end{itemize}


\subsection{Teste 1}
O teste 1 foi realizado com sucesso, como pode ser visto na Figura \ref{fig:test1}, o \emph{Maestro} foi capaz de autenticar os três usuários:
``Victor Alicino'', ``Sergio Dias Alicino'' e ``Meyre Alicino'' como observado nos logs.

\begin{figure}[h!]
    \conteudoFigura
    [Elaborado pelo Autor (2025)]
    {0.8}
    {teste1.png}
    {LOGs do \emph{Maestro} após o login dos usuários}
    {fig:test1}
\end{figure}


\subsection{Teste 2}
\begin{figure}[h!]
    \conteudoFigura
    [Elaborado pelo Autor (2025)]
    {0.5}
    {teste2.png}
    {QR Code gerado pelo \emph{Maestro}}
    {fig:test2}
\end{figure}

Como mostrado na Figura~\ref{fig:maestro-qr}, na aba ``Preview'' do Insomnia, é possível visualizar o QR Code que foi gerado pelo \emph{Maestro} após a
requisição. Lendo o código de barras, se obtém o seguinte JWT:\@
\begin{lstlisting}
    eyJhbGciOiJIUzI1NiIsInR5cCI6IkpXVCJ9.eyJzZXJ2ZXJfaWQiOiJhMDVk
    MmRiMS1kNDBiLTRiZGItYWI0Ny1mMjQzNDY4ZjI3YjciLCJkZXZpY2VfaWQiO
    iJlM2U5OTA2Zi1mY2IwLTExZWYtOGRlZC0wMDFhN2RkYTcxMGEifQ.TmXgA6H
    7rrVV3PIim_FJezCzjtCF4jFpKa-b2uYBTpg
\end{lstlisting}

Que ao ser decodificado com a chave secreta do \emph{Maestro} é obtido o JSON mostrado na Figura~\ref{fig:test2-result}.
\begin{figure}[h!]
    \conteudoFigura
    [Elaborado pelo Autor (2025)]
    {0.2}
    {qr_code_result.png}
    {JWT do QR Code decodificado}
    {fig:test2-result}
\end{figure}
Onde ``server\_id'' é o id do servidor \emph{Opus} (o mesmo que foi utilizado na requisição), ``grant\_until'' é a data de expiração do acesso do 
visitante, ``device'' e ``state'' são dados passados para o \emph{Conductor} para que ele saiba qual tela mostrar e em que estado o dispositivo
se encontra.

Como as informações estão corretas, o teste foi considerado um sucesso.

\subsection{Teste 3}
Seguindo o roteiro do teste 3, os usuários 1, 2 e 3 serão respectivamente ``Victor'', ``Sergio'' e ``Meyre''; o dispositivo a ser controlado
será o ``Ar Condicionado Teste'' com id ``e3e9906f-fcb0-11ef-8ded-001a7dda710a''.

Os \textbf{passos 1 e 2} do roteiro foram realizados com sucesso, como é possível ver na figura~\ref{fig:test3-1}, o log do \emph{Maestro}, na primeira linha,
o usuário ``Victor'' escaneou com sucesso o QR Code recebendo o acesso de visitante ao dispositivo ``Ar Condicionado Teste'', e logo em seguida,
nas próximas linhas é possível ver que o usuário tentou controlar o dispositivo.

\begin{figure}[h!]
    \conteudoFigura
    [Elaborado pelo Autor (2025)]
    {0.35}
    {teste3_2.png}
    {LOGs do \emph{Maestro} 1}
    {fig:test3-1}
\end{figure}

Já na figura~\ref{fig:test3-2}, é possível ver que o \emph{Opus} recebeu a requisição de controle do dispositivo e a executou com sucesso.

\begin{figure}[h!]
    \conteudoFigura
    [Elaborado pelo Autor (2025)]
    {0.4}
    {teste3_1.png}
    {LOGs do \emph{Opus} 1}
    {fig:test3-2}
\end{figure}

Prosseguindo para os \textbf{passos 3 e 4}, o usuário ``Sergio'' escaneou o QR Code e tentou controlar o dispositivo, 
o que também foi feito com sucesso como mostra os logs do \emph{Maestro} na Figura~\ref{fig:test3-3} e do \emph{Opus} na Figura~\ref{fig:test3-4}.

\begin{figure}[h!]
    \conteudoFigura
    [Elaborado pelo Autor (2025)]
    {0.35}
    {teste3_3.png}
    {LOGs do \emph{Maestro} 2}
    {fig:test3-3}
\end{figure}

\begin{figure}[h!]
    \conteudoFigura
    [Elaborado pelo Autor (2025)]
    {0.4}
    {teste3_4.png}
    {LOGs do \emph{Opus} 2}
    {fig:test3-4}
\end{figure}

O usuário ``Sergio'' é reconhecido como visitante, pois não tem permissões de administrador, mas está temporariamente autorizado a controlar o dispositivo.

Seguindo para o \textbf{passo 5}, o usuário ``Meyre'' escaneou o QR Code também com sucesso.

Logo em seguida no \textbf{passo 6}, o usuário ``Sergio'' tentou controlar o dispositivo, o que não foi possível, como esperado, recebendo do \emph{Conductor}
a seguinte mensagem: ``Você não tem mais acesso a esse dispositivo'', como mostrado na figura \ref{fig:test3-5}.

\begin{figure}[h!]
    \conteudoFigura
    [Elaborado pelo Autor (2025)]
    {0.15}
    {conductor_device_guess_lost_access.png}
    {Mensagem de aviso de perda de acesso no \emph{Conductor}}
    {fig:test3-5}
\end{figure}

E por fim, no \textbf{passo 7} o usuário ``Meyre'' tentou controlar o dispositivo, o que foi feito com sucesso como mostra os logs do \emph{Opus}
na Figura \ref{fig:test3-6}.

\begin{figure}[h!]
    \conteudoFigura
    [Elaborado pelo Autor (2025)]
    {0.4}
    {teste3_6.png}
    {LOGs do \emph{Opus} 3}
    {fig:test3-6}
\end{figure}

\subsection{Conclusão Parcial}

Os três testes realizados, mostram que o sistema \textbf{Opus} satisfaz os seguintes objetivos específicos:
\begin{itemize}
    \item Propor e implementar um sistema de gerenciamento para edifícios inteligentes:
        provado pelos testes 1, 2 e 3, o sistema \textbf{Opus} foi capaz de permitir acesso a um
        dispositivo que controla um Ar-Condicionado, permitindo aos usuários alterar o ambiente do 
        edifício de acordo com suas necessidades e vontades.
    \item Tornar este sistema acessível a múltiplos usuários como visitantes:
        provado pelo teste 3, o sistema \textbf{Opus} foi capaz de permitir que múltiplos usuários
        pudessem controlar um dispositivo, seguindo as regras de acesso definidas no capítulo~\ref{chap:multiusuarios}.
\end{itemize}

Sendo assim, o sistema \textbf{Opus} foi capaz de atingir os objetivos específicos propostos, 
permitindo o controle de dispositivos IoT por múltiplos usuários.


