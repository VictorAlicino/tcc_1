\chapter{Testes e Resultados}
\label{testes_e_resultados}

Para testar a viabilidade do sistema proposto, foi desenvolvido as três partes principais do sistema: o Servidor Local, 
o Servidor Remoto e a Interface de Usuário, porém, interfaces para comandos de manipulações do banco de dados do Servidor Local não foram
implementadas, como definir quais níveis de acesso podem controlar quais dispositivos, sendo necessária a manipulação direta do banco de dados.

\section{Testes}

O objetivo principal dos testes foi verificar se o sistema é capaz de aceitar comandos a um dispositivo, sem a necessidade de associar todos os 
usuários ao Servidor Local onde o dispositivo se encontra.

\subsection{Materiais}

Para a realização dos testes, foram utilizados os seguintes materiais:

\textbf{Servidor Local}: no Servidor Local, para executar o \emph{Opus} foi utilizado um computador com o sistema operacional Windows 10,
um Raspberry Pi 3 na mesma rede do computador executando um broker MQTT Mosquitto e um dispositivo emissor de infravermelho inteligente,
modificado para suportar o firmware Tasmota. O dispositivo emissor de infravermelho inteligente foi configurado para se comunicar com o broker
MQTT do Raspberry Pi 3 e configurado no \emph{Opus} para controlar um ar-condicionado.

\textbf{Servidor Remoto}: no Servidor Remoto, para executar o \emph{Maestro} foi utilizado uma instância na OCI (Oracle Cloud Infrastructure)
com o sistema operacional Ubuntu 22.04.5 LTS, um broker MQTT Mosquitto também foi configurado na instância.

\textbf{Interface de Usuário}: para a Interface de Usuário, foram utilizados três smartphones com o sistema operacional Android para executar
o aplicativo \emph{Conductor}, um para cada usuário.

\subsection{Testes}

Tanto o \emph{Opus} quanto o \emph{Maestro} produzem logs de suas atividades, que foram utilizados para verificar se os comandos enviados
pela Interface de Usuário foram recebidos e processados corretamente.

\subsubsection{Teste 1}
O primeiro teste consistiu em verificar se o \emph{Maestro} é capaz de suportar três usuários utilizando a autenticação do Google enviada pelo
\emph{Conductor}.

\begin{figure}[h!]
    \conteudoFigura
    [Elaborado pelo Autor (2025)]
    {0.6}
    {teste1.png}
    {LOGs do \emph{Maestro} após o login dos usuários}
    {fig:test1}
\end{figure}

Como pode ser visto na Figura \ref{fig:test1}, o \emph{Maestro} foi capaz de autenticar os três usuários.

\subsubsection{Teste 2}
O segundo teste consistiu em verificar se o \emph{Maestro} é capaz de gerar um QR Code para um dispositivo. Como não há uma interface para 
acessar essa rota no \emph{Maestro}, foi necessário acessar diretamente a rota por um software de requisições HTTP.

Utilizando o software Insomnia, foi feita uma requisição GET para a rota:
\begin{lstlisting}
    /opus_server/qr_code/a05d2db1-d40b-4bdb-ab47-f243468f27b7/
    e3e9906f-fcb0-11ef-8ded-001a7dda710a
\end{lstlisting}
onde ``a05d2db1-d40b-4bdb-ab47-f243468f27b7'' 
é o ID do Servidor \emph{Opus} utilizado no teste e ``e3e9906f-fcb0-11ef-8ded-001a7dda710a'' é o ID do dispositivo com Tasmota no Servidor Local.

\begin{figure}[h!]
    \conteudoFigura
    [Elaborado pelo Autor (2025)]
    {0.5}
    {teste2.png}
    {QR Code gerado pelo \emph{Maestro}}
    {fig:test2}
\end{figure}

Como mostrado na Figura \ref{fig:maestro-qr}, o \emph{Maestro} foi capaz de gerar o QR Code corretamente.

\subsubsection{Teste 3}

O último teste consistiu em verificar se o \emph{Maestro} é capaz de gerenciar o acesso de um dispositivo para usuários visitantes, seguindo os limites
estabelecidos no capítulo~\ref{chap:multiusuarios} 

O teste consistiu em primeiro o usuário ``Victor'' escanear o QR Code gerado no teste 2 e então tentar controlar o dispositivo, o que foi feito com sucesso
como mostra os logs do \emph{Maestro} na Figura \ref{fig:test3-1} e do \emph{Opus} na Figura \ref{fig:test3-2}.

\begin{figure}[h!]
    \conteudoFigura
    [Elaborado pelo Autor (2025)]
    {0.35}
    {teste3_2.png}
    {LOGs do \emph{Maestro} 1}
    {fig:test3-1}
\end{figure}

\begin{figure}[h!]
    \conteudoFigura
    [Elaborado pelo Autor (2025)]
    {0.4}
    {teste3_1.png}
    {LOGs do \emph{Opus} 1}
    {fig:test3-2}
\end{figure}

Em seguida, o usuário ``Sergio'' escaneou o QR Code e tentou controlar o dispositivo, o que também foi feito com sucesso como mostra os
logs do \emph{Maestro} na Figura \ref{fig:test3-3} e do \emph{Opus} na Figura \ref{fig:test3-4}.

\begin{figure}[h!]
    \conteudoFigura
    [Elaborado pelo Autor (2025)]
    {0.35}
    {teste3_3.png}
    {LOGs do \emph{Maestro} 2}
    {fig:test3-3}
\end{figure}

\begin{figure}[h!]
    \conteudoFigura
    [Elaborado pelo Autor (2025)]
    {0.4}
    {teste3_4.png}
    {LOGs do \emph{Opus} 2}
    {fig:test3-4}
\end{figure}

Logo em seguida, o usuário ``Meyre'' escaneou o QR Code e o usuário ``Sergio'' tentou controlar o dispositivo, o que não foi possível, como esperado.

E por fim o usuário ``Meyre'' tentou controlar o dispositivo, o que foi feito com sucesso como mostra os logs do \emph{Opus} na Figura \ref{fig:test3-5}.

\begin{figure}[h!]
    \conteudoFigura
    [Elaborado pelo Autor (2025)]
    {0.4}
    {teste3_6.png}
    {LOGs do \emph{Opus} 3}
    {fig:test3-5}
\end{figure}

\section{Resultados}

Os testes realizados mostraram que o sistema é capaz de aceitar comandos a um dispositivo, sem a necessidade de associar todos os usuários ao
Servidor Local onde o dispositivo se encontra, ou seja, o sistema é capaz de gerenciar o controle de múltiplos usuários, mesmo sem um cadastro prévio,
a um dispositivo de forma organizada.