\section{Resultados}

% \subsection{Objetivo Geral}
% Este trabalho visa desenvolver um sistema básico que demonstre a viabilidade da operação simultânea de múltiplos usuários
% em um edifício inteligente. A proposta é criar uma solução funcional que atenda a múltiplos usuários em um ambiente compartilhado. 
% 
% % Objetivos Específicos	
% \subsection{Objetivos Específicos}
% 
% Para alcançar o objetivo do trabalho, é proposto uma aplicação simplificada que permita entender a viabilidade do uso de um 
% sistema de edifício inteligente por diversos usuários, focando no controle das unidades de climatização.
% 
% Dentre os principais objetivos específicos destacam-se:
% 
% \begin{itemize}
%     \item Realizar uma pesquisa bibliográfica sobre edifícios inteligentes e suas demandas;
%     \item Definir requisitos para um sistema de gerenciamento de um edifício inteligente;
%     \item Propor e implementar um sistema de gerenciamento para edifícios inteligentes;
%     \item Tornar este sistema acessível a múltiplos usuários sem necessidade de cadastro prévio;
% \end{itemize}

Ao longo da realização deste trabalho, os seguintes objetivos específicos foram alcançados:

\paragraph{Definir requisitos para um sistema de gerenciamento de um edifício inteligente:}
Com base na revisão bibliográfica realizada, foram definidos os requisitos do sistema apresentado no capítulo~\ref{} 