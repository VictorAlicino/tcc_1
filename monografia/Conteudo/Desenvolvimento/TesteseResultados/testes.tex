\section{Testes}

Os testes buscaram validar se o objetivo específico ``Propor e implementar um sistema de gerenciamento para edifícios inteligentes'' foi atingido,
assim como  ``Tornar este sistema acessível a múltiplos usuários como visitantes'', para isso foi necessário executar três testes,
com objetivo de: 
\begin{enumerate}
    \item Verificar se o sistema é capaz de manter mais de um usuário;
    \item Verificar se o sistema é capaz de gerar a chave (QR Code) para um dispositivo acessível para múltiplos usuários;
    \item Verificar se o sistema é capaz de permitir o controle do dispositivo para múltiplos usuários seguindo os limites especificados.
\end{enumerate}

Tanto o \emph{Opus} quanto o \emph{Maestro} produzem logs de suas atividades, que foram utilizados para verificar se os comandos enviados
pela Interface de Usuário foram recebidos e processados corretamente.

\subsection{Teste 1}
O primeiro teste consistiu em verificar se o \emph{Maestro} é capaz de suportar três usuários utilizando a autenticação do Google enviada pelo
\emph{Conductor}.

Ao abrir o aplicativo \emph{Conductor}, o usuário é recebido pela tela de login do aplicativo, como mostrado na subseção \ref{subsec:conductor-auth},
o teste consistiu em autenticar três usuários diferentes, ``Victor Alicino'', ``Sergio Dias Alicino'' e ``Meyre Alicino''.

Ao enviar os dados da conta Google do usuário por uma requisição \emph{POST} na rota \lstinline{/auth/conductor/login}, o
servidor \emph{Maestro} deve responder com um token de autenticação para cada usuário, que será usado nas requisições dos próximos testes,
o teste é considerado um sucesso, se ao verificar os logs do \emph{Maestro} for encontrado o log de nível \emph{DEBUG} notificando que o
usuário foi autenticado.

\subsection{Teste 2}
O segundo teste consistiu em verificar se o \emph{Maestro} é capaz de gerar um QR Code para um dispositivo. 
Como não há uma interface de usuário para acessar essa rota no \emph{Maestro}, foi necessário acessar diretamente a rota por um software de
requisições HTTP e incluir o token de autenticação de um usuário com permissão de administrador na requisição.
O servidor \emph{Opus} utilizado possui no banco de dados do \emph{Maestro} o id ``\lstinline{a05d2db1-d40b-4bdb-ab47-f243468f27b7}'' e o
dispositivo ar-condicionado está cadastrado dentro desse servidor \emph{Opus} com o id ``\lstinline{e3e9906f-fcb0-11ef-8ded-001a7dda710a}''.

A rota para gerar o QR Code é \lstinline{/opus_server/qr_code/<server_id>/<device_id>}, onde ``server\_id'' é o id do servidor \emph{Opus} e
``device\_id'' é o id do dispositivo no servidor local, utilizando o software Insomnia, foi feita uma requisição GET para esta rota com os seguintes 
parâmetros:
\begin{lstlisting}
    /opus_server/qr_code/a05d2db1-d40b-4bdb-ab47-f243468f27b7/
    e3e9906f-fcb0-11ef-8ded-001a7dda710a
\end{lstlisting}

Em caso de sucesso do teste, é esperado que o \emph{Maestro} retorne um QR Code carregando um token JWT que será utilizado 
pelo \emph{Opus} para controlar o dispositivo. 

\subsection{Teste 3}

O terceiro teste, foi focado em verificar o funcionamento do controle de dispositivos por múltiplos usuários, seguindo as regras de acesso
definidas no capítulo~\ref{chap:multiusuarios}. Para isso, este teste contou com um roteiro:
\begin{enumerate}
    \item O primeiro usuário, escaneia utilizando o aplicativo \emph{Conductor} o QR Code gerado no teste 2;
    \item O primeiro usuário então irá tentar controlar o dispositivo, realizando algum comando como ligar, desligar ou alterar a temperatura;
    \item O segundo usuário, irá escanear utilizando o aplicativo \emph{Conductor} o mesmo QR Code;
    \item O segundo usuário então irá tentar controlar o dispositivo, realizando algum comando como ligar, desligar ou alterar a temperatura;
    \item O terceiro usuário, irá escanear utilizando o aplicativo \emph{Conductor} o mesmo QR Code;
    \item O segundo usuário, tentará controlar o dispositivo após o terceiro usuário ter escaneado o QR Code.
    \item O terceiro usuário então irá tentar controlar o dispositivo, realizando algum comando como ligar, desligar ou alterar a temperatura;
\end{enumerate}

Neste teste, é esperado que no item 6, o segundo usuário não consiga controlar o dispositivo após o terceiro usuário ter escaneado o QR Code,
isso deverá acontecer, pois o servidor \emph{Maestro} irá trocar o usuário visitante sempre que um novo (usuário) escanear o QR Code. O servidor não
deve permitir que mais de um visitante possua controle de um dispositivo ao mesmo tempo, mas deve garantir que todos os visitantes que escanearam o QR Code
possam controlar o dispositivo.