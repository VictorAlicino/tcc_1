\chapter{Introdução} 
\label{cap:introducao}

% Introdução:

% Objetivo do trabalho
% Contextualização do problema
% Estrutura do trabalho (explicando brevemente o que será abordado em cada seção)

O tema cidades inteligente vem ganhando muita atenção nos últimos anos, tais cidades tem por estratégia gerar uma melhor 
qualidade de vida para o cidadão e um desenvolvimento econômico e social mais sustentável através do uso de tecnologias da 
informação e comunicação \cite{cetic}, um membro das cidades inteligente são edifícios inteligentes. Indivíduos passam grande 
parte da sua vida dentro de edifícios, crescem, estudam e se desenvolvem neles \cite{art1} \cite{noauthor_how_nodate}, ao levar 
em conta a importância que essas estruturas têm no cotidiano, fica claro seu papel nas cidades inteligentes. 
Desta forma, um dos passos para a realização das cidades inteligentes é a modernização dos edifícios, alinhá-los com as propostas
estabelecidas de melhorar a qualidade de ida do ocupante e contribuir com um desenvolvimento sustentável.

Um edifício se torna inteligente através da adição de uma série de dispositivos com a finalidade de monitorar e controlar o ambiente,
podem ser citados os seguintes dispositivos:
\begin{itemize}
    \item Sensores;
    \item Atuadores;
    \item Controladores;
    \item Unidade Central;
    \item Interface;
    \item Rede;
    \item Medidor inteligente.
\end{itemize}
\cite{Morvaj2011} Suas finalidades são de operar em conjunto para a realização de tomadas de decisões, sejam elas automatizadas ou 
com interferência humana.
Tais dispositivos precisam ser gerenciados por um sistema central que atende as demandas que tornam um edifício inteligente. 

Quando o assunto são casas inteligentes, sistemas como este já são realidade e estão consolidados no mercado; Segundo a plataforma
Statista, aproximadamente 91 milhões de \emph{Smart Speakers} {(}do inglês, caixa de som inteligente, são dispositivos que atuam 
duplamente como caixas de som e também como interface de interação humana com algum assistente virtual{)} foram instalados nos 
Estados Unidos no ano de 2021 \cite{statista1}, dos quais desde 2020, 60\% dos \emph{Smart Speakers} já se conectavam a plataforma
Amazon Alexa \cite{statista2}. A plataforma Amazon Alexa provê ao usuário uma maneira fácil de se conectar a vários dispositivos de
casa inteligente e controlar-los todos através de um só lugar, assim como a plataforma de código aberto Home Assistant, um sistema de
automação residencial \cite{assistant_home_nodate} que oferece várias opções de configurações e personalização e possui atualmente
255 mil instalações ativas \cite{home_assistant1}. Ambos os softwares foram criados com o indivíduo e uso individual em mente, não sendo
boas opções para o uso em edifício inteligentes onde mais de um indivíduo pode querer interagir com o sistema do edifício.

