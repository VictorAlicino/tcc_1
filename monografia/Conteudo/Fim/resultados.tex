\chapter{Resultados}
\label{resultados}

\section{Comunicação entre \emph{Opus} e \emph{Maestro}}
\label{resultados:comunicacao}
A comunicação entre Servidor Local (\emph{Opus}) e Servidor Remoto (\emph{Maestro}) foi implementada com sucesso.
A troca de mensagens que utiliza o protocolo MQTT com um \emph{broker} hospedado na nuvem ocorreu como esperado,
a monitoração entre a troca de mensagens na inicialização de um servidor \emph{Opus} pode ser vista na Figura \ref{fig:resultados1}.
\begin{figure}[h!]
    \conteudoFigura
    [Elaborado pelo Autor (2024)]
    {0.5}
    {resultados1.png}
    {Comunicação entre \emph{Opus} e \emph{Maestro}}
    {fig:resultados1}
\end{figure}
O Servidor Local (\emph{Opus}) enviou no tópico \lstinline{maestro/login}, tópico onde o Servidor Remoto (\emph{Maestro}) escuta por conexões
de Servidores Locais, os dados com o qual ele se desejava conectar, o Servidor Remoto (\emph{Maestro}) respondeu com uma mensagem de confirmação
no tópico escolhido \emph{Opus} (\lstinline{opus-server-5be2/callback/cloud_manager}) contendo o nome do servidor \emph{Opus}, seu ID no banco de dados
e como este servidor já estava cadastrado, o \emph{Maestro} enviou junto a lista de usuários associados a este \emph{Opus}, onde podemos ver
usuários com ``\lstinline{role}'' 0 e 1, onde 0 é o administrador e 1 é o usuário comum. 
