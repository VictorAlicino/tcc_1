\chapter{Conclusão}
\label{conclusao}

% \section{Objetivos}
% \subsection{Objetivo Geral}
% Este trabalho visa desenvolver um sistema básico que demonstre a viabilidade da operação simultânea de múltiplos usuários
% em um edifício inteligente. A proposta é criar uma solução funcional que atenda a múltiplos usuários em um ambiente compartilhado. 
% 
% % Objetivos Específicos	
% \subsection{Objetivos Específicos}
% 
% Para alcançar o objetivo do trabalho, é proposto uma aplicação simplificada que permita entender a viabilidade do uso de um 
% sistema de edifício inteligente por diversos usuários, focando no controle das unidades de climatização.
% 
% Dentre os principais objetivos específicos destacam-se:
% 
% \begin{itemize}
%     \item Realizar uma pesquisa bibliográfica sobre edifícios inteligentes e suas demandas;
%     \item Definir requisitos para um sistema de gerenciamento de um edifício inteligente;
%     \item Propor e implementar um sistema de gerenciamento para edifícios inteligentes;
%     \item Tornar este sistema acessível a múltiplos usuários como visitantes;
% \end{itemize}

% Pequeno compilado dos resultados ------------------------------------------------------------------
O objetivo geral do trabalho pôde ser alcançado através da conclusão dos objetivos específicos, onde ficou claro que é possível um sistema
multiusuário voltado para edifícios inteligentes.
% Servidor Local
O conceito de servidor local obteve sucesso no controle de dispositivos dentro da mesma rede, concentrando toda a responsabilidade do mantimento
dos dispositivos dentro da rede local, o que garante um controle maior dos dados para o usuário que implementar o sistema \textbf{Opus} no seu 
edifício, o esquema interligado de \emph{Drivers} e \emph{Interfaces} permitiu a criação de um sistema modular, tornando o 
sistema \textbf{Opus} flexível e adaptável a tecnologias presentes e futuras no mercado.
% Servidor Remoto
A entidade servidor remoto atuou muito bem como um intermediário entre o servidor local e o usuário, expondo rotas que possibilitaram o acesso da 
interface de usuário aos servidores remotos concentrando o fluxo de mensagens em um só lugar, também mantendo o controle e responsabilidade 
quanto a autenticação dos usuários, o que dá ao sistema todo uma maior garantia de integridade na autenticidade de quem seja um usuário.
% Interface de Usuário
A interface de usuário em forma de aplicativo móvel, como um envólucro das rotas do servidor remoto para o usuário final cumpriu seu papel
expondo para os usuários a interação com o sistema de forma intuitiva e também sendo um link confiável para a autenticação através do Google,
o que dá ao servidor remoto garantia da autenticidade do usuário.
% Multiusuários
Algumas adições ao sistema \textbf{Opus} foram necessárias para que o sistema atendesse o objetivo geral de permitir o controle de múltiplos usuários,
mesmo aqueles não autorizados previamente por permissões de visitantes, das quais se destaca o uso de um código de barras bidimensional (QR Code) para 
que o aplicativo móvel pudesse fazer o pedido de acesso de visitante ao servidor remoto, possibilitando que um usuário não autorizado
previamente pudesse controlar um dispositivo por um tempo limitado, garantindo assim o título de sistema multiusuário.

% Contribuições do seu estudo -----------------------------------------------------------------------
O estudo de um sistema multiusuário para edifícios inteligentes, como o \textbf{Opus}, contribui para a área de automação de edifícios,
mostrando que é possível a integração de dispositivos IoT acessíveis a edifícios que possuem áreas de uso coletivo e rotatividade de usuários,
como escritórios e coworking.

% Possibilidades de novos estudos -------------------------------------------------------------------
\chapter{Trabalhos Futuros}
\label{trabalhos_futuros}

Provada a arquitetura proposta, é possível expandir o sistema para atender a mais dispositivos, integrando mais partes de um edifício, abrangendo uma
maior quantidade de hardwares disponíveis no mercado.
Recomenda-se para trabalhos futuros:
\begin{itemize}
    \item \textbf{Integração com mais dispositivos}: A integração com mais dispositivos pode ser feita através da criação de novos \emph{Drivers}, 
    abrangendo mais hardwares disponíveis no mercado, como controle de portas, equipamentos de projeção, controle de cortinas, entre outros.
    \item \textbf{Integração com mais protocolos}: A integração com mais protocolos pode ser feita através da criação de novas \emph{Interfaces},
    é teóricamente possível incluir de automação que não sejam necessáriamente IoT ao \emph{Opus}, esses dispositivos utilizam outros protocolos de comunicação,
    como mencionados no capítulo 4 (\ref{tecnologias}), como o BACnet, KNX e ZigBee, que podem ser incluídos ao se criar \emph{Interfaces} capazes de interagir
    com esses protocolos.
    \item \textbf{Implementação de um hierarquia à definir pelo administrador}: A implementação de uma hierarquia de usuários pode ser feita através da criação
    de mais ``\lstinline{roles}'', este campo do banco de dados é um inteiro que conta a maior hierarquia como 0 (administrador), sendo assim, é possível criar
    mais níveis de hierarquia, como um usuário que pode controlar somente alguns dispositivos em uma sala, de acordo com a necessidade do administrador.
\end{itemize}

% Considerações Finais ------------------------------------------------------------------------------
\section{Considerações Finais}
Por fim, o desenvolvimento de um sistema multiusuário para edifícios inteligentes se provou um desafio multidisciplinar,
onde foi não somente necessário o conhecimento básico da linguagem de programação escolhida, mas também suas pequenas
particularidades para executar os requisitos específicos do projeto, também se fez necessário o conhecimento em Redes de Computadores 
para organizar as conexões entre os servidores e configurar as máquinas de testes para atuarem da forma que a arquitetura esperava.
Dado todos os desafios, foi possível testar a arquitetura proposta e verificar que é possível o controle de múltiplos usuários
a um mesmo dispositivo em um sistema de edifício inteligente.