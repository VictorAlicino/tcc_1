\chapter{Trabalhos Futuros}
\label{trabalhos_futuros}

Provada a arquitetura proposta, é possível expandir o sistema para atender a mais dispositivos, integrando mais partes de um edifício, abrangendo uma
maior quantidade de hardwares disponíveis no mercado.
Recomenda-se para trabalhos futuros:
\begin{itemize}
    \item \textbf{Integração com mais dispositivos}: A integração com mais dispositivos pode ser feita através da criação de novos \emph{Drivers}, 
    abrangendo mais hardwares disponíveis no mercado, como controle de portas, equipamentos de projeção, controle de cortinas, entre outros.
    \item \textbf{Integração com mais protocolos}: A integração com mais protocolos pode ser feita através da criação de novas \emph{Interfaces},
    é teóricamente possível incluir de automação que não sejam necessáriamente IoT ao \emph{Opus}, esses dispositivos utilizam outros protocolos de comunicação,
    como mencionados no capítulo 4 (\ref{tecnologias}), como o BACnet, KNX e ZigBee, que podem ser incluídos ao se criar \emph{Interfaces} capazes de interagir
    com esses protocolos.
    \item \textbf{Implementação de um hierarquia a definir pelo administrador}: A implementação de uma hierarquia de usuários pode ser feita através da criação
    de mais ``\lstinline{roles}'', este campo do banco de dados é um inteiro que conta a maior hierarquia como 0 (administrador), sendo assim, é possível criar
    mais níveis de hierarquia, como um usuário que pode controlar somente alguns dispositivos em uma sala, de acordo com a necessidade do administrador.
\end{itemize}